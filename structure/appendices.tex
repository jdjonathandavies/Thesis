\chapter{Thermal Coincidence Trigger}
\label{chap:appendix:trigger}

The ARA trigger system looks at the rate at which $N$ of $M$ antennas have power above a threshold in a time window $t$. A power threshold is set for each antenna which results in a rate of threshold passing $R_{single}$ from thermal noise. The thresholds are set such that they acheive the same $R_{single}$ rate for each of the antennas contributing to the trigger. This section will describe how the global trigger rate (the rate at which the trigger condition is met) can be determined from the singles rate $R_{single}$.



\section{Single antenna, $N$ hits}
Let us first consider the case of a single antenna which has ``hits'' at the rate $\Gamma$, where a hit is the power received in the antenna crossing the threshold. The rate for $N$ hits is then given by the hit rate multiplied by the probability that the hit is followed by $N-1$ additional hits within the coincidence window $t$:

\begin{equation}
  \Gamma_{N} = \Gamma P_{N-1}
\end{equation}

The probability of $N-1$ hits, $P_{N-1}$, is given by Poisson statistics. The expected number of hits in a coincidence window $t$ is:

\begin{equation}
  \mu = \Gamma t\\
\end{equation}
hence
\begin{equation}
  P_{N} = e^{-\mu}\frac{\mu^{N}}{N!}
\end{equation}
so we have
\begin{equation}
  \Gamma_{N} = \Gamma e^{-\mu}\frac{\mu^{N-1}}{(N-1)!}
\end{equation}

\section{$M$ antennas, $N$ with one or more hits}

The rate for single hits among $M$ antennas is
\begin{equation}
  \Gamma_{1} = M\Gamma
\end{equation}

The rate at which we have $N$ out of $M$ antennas with one or more hits is the rate at which single hits occur ($\Gamma$) multiplied by the number of ways of choosing $N-1$ antennas, weighted by the probability that $N-1$ antennas have one or more hits and the rest have none. The probability for zero hits in an antenna is:

\begin{equation}
  P_{0} = e^{-\mu}
\end{equation}

and the probability for non-zero hits is therefore $1-P_{0} = 1-e^{-\mu}$. Combining the combinatoric term (from the ways of choosing $N-1$ antennas) and the probability factors we arrive at:

\begin{equation}
  \begin{split}
  P_{N-1} &= \frac{(M-1)!}{(M-N)!(N-1)!}(1-P_{0})^{N-1}(P_{0})^{M-N}\\
  &= C^{N-1}_{M-1}  (1-P_{0})^{N-1}(\frac{1}{P_{0}})^{N-1}(P_{0})^{M-1}\\
  &= C^{N-1}_{M-1}  (\frac{1}{P_{0}}-1)^{N-1}(P_{0})^{M-1}\\
  &= C^{N-1}_{M-1}  (e^{\mu}-1)^{N-1}(P_{0})^{M-1}\\
  \end{split}
\end{equation}


So the rate at which the trigger condition is met is then:

\begin{equation}
  \begin{split}
    \Gamma_{N} = M \Gamma P_{N-1} &= M \Gamma C^{N-1}_{M-1} (e^{\mu}-1)^{N-1}(P_{0})^{M-1}
  \end{split}
\end{equation}

The trigger condition is also met when more than $N$ out of $M$ antennas, so a full treatment would sum the contributions from $N$ of $M$, $N+1$ of $M$, ..., $M$ of $M$ antennas. We will only consider the $N$ of $M$ term for now, as the other terms can be found by changing the value of $N$.

If the hits are rare then we have $\mu \ll 1$, then $e^{\mu}-1 \rightarrow \mu = \Gamma t$ and $P_{0} \rightarrow 1$. So we have:

\begin{equation}
  \begin{split}
    \Gamma_{N} &= M \Gamma \frac{(M-1)!}{(M-N)!(N-1)!} \Gamma^{N-1} t^{N-1} \\
    &= N \frac{M(M-1)!}{(M-N)!N(N-1)!} \Gamma^{N} t^{N-1}\\
    &= N C^{N}_{M}     \Gamma^{N} t^{N-1}\\
  \end{split}
\end{equation}

Now using the nomenclature from the ARA trigger systems we have $R_{global} = \Gamma_{N}$ and that $\Gamma = R_{single}$ we have:

\begin{equation}
  R_{global} = NC^{N}_{M}R^{N}_{single}t^{N-1}\\
  \label{eq:ARA-Trigger-Rate}
\end{equation}


It is clear that as the number of coincident antennas $N$ increases the rate falls by a factor of $R_{single}t$. In the case of the TestBed the singles rate $R_{single} \sim 10 \kilo\hertz$ and the time window $t = 100 \nano\second$ so the contributions fall by a factor of a thousand. \EquationRef{eq:ARA-Trigger-Rate} is therefore a good leading order approximation to the trigger rate expected from thermal noise in ARA stations using such a coincidence trigger.
