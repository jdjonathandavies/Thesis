\chapter{Ultra-high Energy Astro-particle Physics}
\label{chap:uhe-app}


Astro-particle physics encompasses detection of a wide range of particles produced in the Universe at large. The long standing goal of this field is to understand the high energy Universe through observations of these particles, however their detection is extremely challenging. As detector technologies improved astrophysical observations through optical telescopes have been complemented by other frequencies of light including gamma rays, X-rays and infrared frequencies. Recent experiments such as the HESS \cite{Hinton2004331} and VERITAS \cite{Holder:2008ux} observatories have extended sensitivity at high energies, which will be further improved with the construction of the Cherenkov Telescope Array \cite{2011ExA....32..193A}.

It was realised that electromagnetic radiation was not the only resource available to aid our understanding of astrophysical objects and phenomena. At present there are a number of other astrophysical messengers available to scientists each carrying new and often complementary information about the near and distant Universe. Very quickly it became as interesting to understand these particles themselves and the physics that governs their creation and interactions. The study of charged particles incident upon the Earth's atmosphere heralded a new era in physics in which a myriad of composite and elementary particles were discovered and opened a new discipline in the form of particle physics.

Since the days of Victor Hess and his electroscope experiments \cite{HessNobelLectures} much has been learned about the flux and energy of cosmic rays\footnote{In this document the term ``cosmic ray'' will refer to protons or atomic nuclei, not to gamma rays or neutrinos} incident on the Earth and many experiments have observed high energy particles of astrophysical origin. By some mechanism distant sources are able to accelerate charged particles to energies in excess of \EeV ($10^{18}$\eV), as cosmic rays with these energies have been observed for half a century \cite{Linsley1963}. These ultra high energy (UHE) cosmic rays can impact on a stationary targets, for example a nucleus in the atmosphere, interacting with centre of mass energies in excess of $\sim100\TeV$, an order of magnitude higher than achievable with current accelerator technology. For this reason they provide a fascinating glimpse into particle physics in a regime inaccessible to scientists through conventional means. 

UHE cosmic rays and other astrophysical messengers carry information about their sources and provide a powerful means to probe the high energy Universe, facilitating the study of extreme conditions in which to test and inform our understanding of physics in the Universe at large. There are, however, limitations to the information that we can learn from many of the messengers used at present. All but the most energetic charged particles will have their trajectories significantly  bent by magnetic fields encountered in transit to the Earth. The other main limitation for most messengers are the horizon effects that limit the distances they can travel. UHE gamma-rays will pair produce \Pelectron\Ppositron off the cosmic microwave background (CMB) preventing them from travelling large distances. Unbound neutrons are unstable with a proper lifetime of around 15 minutes \footnote{At 100\TeV the neutron lifetime is $\sim 3 $years}, meaning that they will decay in flight producing cosmic rays. Although CMB photons have very low energies the centre of mass energy available when struck by UHE cosmic rays can be sufficient to cause photo-pion production, thus restricting the range of UHE cosmic rays. Neutrinos, on the other hand, do not suffer these horizon effects. Even at ultra-high energies weakly interacting neutrinos have such small cross-sections that they travel effectively unimpeded throughout the Universe. 

The horizon effects that limit the information that we can obtain about the most energetic cosmic rays may also produce other messengers detectable on Earth. The interactions with CMB photons are expected to produce a flux of UHE neutrinos which, if detected, would provide vital information that may resolve some of the mysteries surrounding UHE cosmic rays. These cosmic rays will be sufficiently boosted that they will point back toward their sources and, since neutrinos are not bent by magnetic fields, it may be possible to identify UHE cosmic ray sources via the associated UHE neutrinos they produce. Furthermore the association of UHE neutrinos with the UHE cosmic ray flux leads to a number of spectral features that may prove vital in distinguishing between possibilities for their sources and acceleration mechanisms.

There are strong links between the flux of UHE neutrinos and UHE cosmic rays. This chapter will discuss the current understanding of the cosmic ray spectrum, the production mechanisms of UHE neutrinos and their detection.

\section{Cosmic rays}
\label{section:uhe-app:Cosmic-Rays}


\begin{figure}[htpb]
  \centering
  \includegraphics[width=\textwidth]{chapters/uhe-app/cr_AllParticle.eps}
  \caption{The cosmic ray spectrum for all charged particles from \cite{Beringer:1900zz}. The spectrum is multiplied by a factor $E^{2.6}$ to highlight some of the key features.}
  \label{fig:uhe-app:Cosmic-Rays:Spectrum}
\end{figure}


The flux of cosmic rays incident upon the Earth's atmosphere has been measured up to energies of around $10^{20}\eV$. The spectrum, which is shown in \FigureRef{fig:uhe-app:Cosmic-Rays:Spectrum}, is steeply falling and is remarkably well approximated with a power law form $dN/dE \propto E^{-\gamma}$ where $\gamma$ ranges from 2.7 to 3. There are a number of features in the spectrum that are thought to be due to transitions between different classes of source and acceleration mechanisms. The three main features are: the cosmic ray \textit{knee} around $10^{15.5}\eV$, the \textit{ankle} at $3\times10^{18}\eV$, and the \textit{cut-off} above $3\times10^{19}\eV$. The spectral index below the knee is $\gamma=2.7$, steepening to $\gamma=3$ between the knee and ankle, at which it returns to a similar index to below the knee. The flux falls off rapidly such that direct detection above $10^{15}\eV$ is very difficult. At these energies the flux drops below tens of particles per m$^{2}$ per year and with typical direct detection experiments using balloons or satellites, being $\sim$m$^{2}$ in size, the collection of large samples is very difficult.


At these energies and above particles can be observed in arrays that sample the extensive air showers produced by cosmic ray interactions in the atmosphere. In addition it is possible to detect cosmic rays by measurements of nitrogen fluorescence from the shower using fluorescence detectors. The Auger experiment \cite{Abraham200450} makes use of both methods, with a ground array of Cherenkov detectors and fluorescence detectors sited on the perimeter. Recently the ANITA \cite{PhysRevLett.105.151101} experiment made observations of cosmic rays via radio pulses originating from the interaction of cosmic ray air showers with the magnetic field in Antarctica. The next flight of ANITA, a balloon-borne experiment, is expected to yield a significantly increased sample of these cosmic rays.


Calculating the primary energy of the cosmic ray in these experiments is challenging for a number of reasons: they involve large numbers of particles and interactions necessitating computer modelling, and the centre of mass energies are well beyond those that can be produced in the laboratory. This means that some extrapolation from experiments such as those at the \LHC is needed, introducing uncertainties in particle populations expected in ground level detectors.

The current thinking in the field is that cosmic rays at and below the energy of the knee are produced in galactic astrophysical sources. The two most popular candidate sources are supernova remnants and binary systems. A popular interpretation is that the feature described as the knee is the result of these sources reaching their maximum acceleration energy. Cosmic ray acceleration mechanisms rely on strong magnetic fields. For heavy nuclei the maximum energy acquired in acceleration will be $Z$ times higher \footnote{Where $Z$ is the charge of the nucleus.}. As we approach the energy of the knee cosmic ray sources cannot accelerate the lightest cosmic rays (protons) to higher energies. The flux is firstly taken over by a population of He nuclei, then by other heavier nuclei in order of charge. This produces a broad feature instead of the sharp one expected from cosmic rays composed of only one type of nucleus.

There are a number of explanations for the feature known as the ankle. One possibility is that it is a result of a higher energy population overtaking a population of lower energy particles, this could be a galactic flux being overtaken by an extra-galactic one. Another possibility is that the dip corresponds to electron positron production caused by interactions between cosmic ray protons and CMB photons\footnote{$\gamma + \Pproton \rightarrow \Pelectron + \Ppositron$} \cite{PhysRevD.74.043005}, which would again rely on the population being extra-galactic in nature due to the large propagation lengths needed for such interactions to take place.


\section{UHE cosmic rays}
\label{section:uhe-app:UHE-Cosmic-Rays}

At energies in excess of $\sim6\times10^{19} \eV$ cosmic ray protons will rapidly loose energy in interactions with CMB photons, enhanced by the $\Delta^{+}$ resonance, shown in \EquationRef{eq:uhe-app:UHECR:GZK}. These interactions lead to a suppression of the high energy tail in the cosmic ray flux. This suppression was first predicted by Greisen \cite{Greisen:1966jv}, Zatsepin and Kuzmin \cite{Zatsepin:1966jv} soon after the discovery of the CMB and is known as the GZK cutoff.



\begin{align}
  \Pproton + \gamma_{CMB} & \rightarrow \Delta^{+} \rightarrow \Ppi + N 
  \label{eq:uhe-app:UHECR:GZK} \\
  &  \rightarrow \Pproton + \Pelectron + \Ppositron 
  \label{eq:uhe-app:UHECR:pair-production}
\end{align}

\noindent Along with the photo-pion production seen in \EquationRef{eq:uhe-app:UHECR:GZK} cosmic rays can produce electron positron pairs as in \EquationRef{eq:uhe-app:UHECR:pair-production}. The threshold for pair production is about $10^{18} \eV$ with a mean free path $\sim1$ Mpc, whereas for photo-pion production the threshold is $6\times10^{19} \eV$ with mean free path $\sim 6$ Mpc. Despite the lower thresholds the energy losses are dominated by photo-pion production as the energy loss per interaction is $\sim 20\%$ compared with only $0.1\%$ for pair production. The GZK cutoff also produces an effective horizon for cosmic ray protons limiting their path length to $\sim100 $Mpc.


In the case that UHE cosmic rays are heavy nuclei photo-disintegration and pair production become important:

\begin{equation}
  A + \gamma_{CMB} & \rightarrow (A-1) + N
  \label{eq:uhe-app:UHECR:photodisintegration}\\
  & \rightarrow (A-2) + 2N
  \label{eq:uhe-app:UHECR:photodisintegration2}\\
  & \rightarrow A + \Pelectron + \Ppositron
  \label{eq:uhe-app:UHECR:pair-production-nucleus}
\end{equation}

\noindent where $A$ is the mass of the nucleus and $N$ is a secondary nucleon. This process leads to a similar suppression in the cosmic ray flux as in the cosmic ray proton case. Nuclei are still able to undergo the GZK process of \EquationRef{eq:uhe-app:UHECR:GZK} both before and after photo-disintegration and pair production, but the nucleons have on average $\frac{1}{A}$ of the nuclei's energy. This results in a shift upward in the threshold for photo-pion production, in the case of iron nuclei to $\sim3.5 \times 10^{21} \eV$.

A suppression in the flux of UHE cosmic rays consistent with the GZK cutoff has been observed in both the Auger \cite{ThePierreAuger:2013eja} and HiRes \cite{Abbasi2008} experiments at $10^{19.5} \eV$ and is shown in \FigureRef{fig:uhe-app:Cosmic-Rays:UHE-Spectrum}. Although UHE cosmic rays at and after the GZK cutoff have been detected the composition in this regime remains uncertain. Measurements of the depth of shower maximum $X_{max}$, a parameter that is strongly correlated with cosmic ray composition, are consistent with a move to heavier nuclei at high energies in the Auger experiment \cite{Abraham:2010yv}.

Air showers produced by primaries heavier than protons can be thought of as a superposition of $A$ showers each with energy $\frac{E}{A}$, where $A$ is the mass of the primary. In such a scenario the depth of shower maximum is reduced to the value of a proton with energy $\frac{E}{A}$ and hence heavier primaries give rise to lower measured values of $X_{max}$. In addition the variation of this quantity, $RMS (X_{max})$, will be reduced due to the presence of $A$ showers, as opposed to a single shower in the case of protons.

\FigureRef{fig:uhe-app:Cosmic-Rays:Auger-Composition} shows the most recent results measuring these parameters with the Auger experiment. It should be noted that the statistics are limited due to the low fluxes at high energies, and the move to smaller values of $X_{max}$ can equally be explained by modifications to the interaction cross section at these energies (which are extrapolated from those measured in particle physics experiments).


\begin{figure}[htpb]
  \centering
  \includegraphics[width=\textwidth]{chapters/uhe-app/cr_HighestEnergy.eps}
  \caption{The cosmic ray spectrum in the UHE regime from \cite{Beringer:1900zz}. The flux is multiplied by $E^{2.6}$ to highlight key features and to aid comparisons with \FigureRef{fig:uhe-app:Cosmic-Rays:Spectrum}. Both HiRes and Auger data show features consistent with the cosmic ray ankle.}
  \label{fig:uhe-app:Cosmic-Rays:UHE-Spectrum}
\end{figure}


\begin{figure}[htpb]
  \centering
  \includegraphics[width=\textwidth]{chapters/uhe-app/Auger-composition.pdf}
  \caption{The UHE cosmic ray composition as measured by Auger from \cite{Abraham:2010yv}. The average depth of shower maximum $\langle X_{max} \rangle$ and the variation in depth of shower maximum $RMS (X_{max})$ are shown as a function of energy from Auger data (points). For comparison the expected values from simulation (lines) are also shown. Both parameters show a trend toward higher mass cosmic rays with increasing energy.}
  \label{fig:uhe-app:Cosmic-Rays:Auger-Composition}
\end{figure}


There remain a number of mysteries surrounding UHE cosmic rays. Due to the GZK cutoff and the associated horizon, their sources must be nearby in cosmological terms. Due to their very high momentum, and hence rigidity, at these energies there should be little deviation from their source by magnetic fields, but to date none have been identified. The energies are so massive that it is very hard to explain how astrophysical objects provide sufficient acceleration, as illustrated in the Hillas plot in \FigureRef{fig:uhe-app:Cosmic-Rays:Hillas-Plot}. Complementary information is required to address some of these unanswered questions, information that may be provided by observations of UHE neutrinos.

\begin{figure} [htpb]
  \includegraphics[width=\largefigwidth]{chapters/uhe-app/HillasPlot.pdf}
  \caption{The Hillas plot from \cite{2011ARA&A..49..119K}. Sources above the red and blue lines are unable to confine (and hence accelerate) iron nuclei to $10^{20}\eV$ and protons to $10^{21}\eV$ via magnetic fields.}
  \label{fig:uhe-app:Cosmic-Rays:Hillas-Plot}
\end{figure}



\section{UHE neutrinos}
\label{sec:uhe-app:UHEN}




Considering the GZK process in \EquationRef{eq:uhe-app:UHECR:GZK} it is clear that subsequent decays of charged pions and neutrons will produce a so called `guaranteed' flux of neutrinos. Beresinsky and Zatsepin \cite{Beresinsky:1969qj} were the first to predict such a flux and they are referred to as (BZ) neutrinos. BZ neutrinos are produced in the following decay of charged pions:

%FIXME

\begin{equation}
  \begin{split}
    \Ppiplus \rightarrow & \APmuon + \Pnum\\
    & \APmuon \rightarrow \APnum + \APelectron + \Pnue\\
    \Pneutron \rightarrow & \Pproton + \Pelectron + \APnue
  \end{split}.
  \label{eq:uhe-app:UHEN:photo-pion-production}
\end{equation}

\noindent The GZK process will produce \Pnue, \Pnum and \APnum with the associated flux  having a ratio of flavour states \Pnue : \Pnum : \Pnut of 1 : 2 : 0. In the case of neutrinos produced in photo-disintegration and neutron decay only \APnue are produced.

Due to the production mechanism there is strong link between the neutrino and cosmic ray spectra. Observing and measuring the properties of BZ neutrinos would provide otherwise unobtainable information regarding the cosmic ray spectrum. For example it would be possible to infer details of the emission spectra of cosmic rays (such as the maximum energy to which they are accelerated) and information about the sources.

An example of a predicted flux is shown in \FigureRef{fig:uhe-app:UHEN:UHEN-Flux}. The flux of \Pnum and \APnum come exclusively from the decay of charged pions from the photo-pion production mechanisms discussed previously. Since the UHE cosmic ray spectrum is steeply falling most of these interactions will occur close to the threshold nucleon energy of $6 \times 10^{19} \eV$ and transfer approximately $5\%$ of the energy available to neutrinos, resulting in a single peak between $10^{18}$ and $10^{19}$ \eV in both the electron and muon neutrino fluxes. The lower energy peak between $10^{16}$ and $10^{17}$ \eV is due to neutron decays producing \APnue. For reference the Waxman-Bachall \cite{Waxman:1998yy} \cite{Bahcall:1999yr} limit is shown on the same axes. This limit is an upper bound on neutrino fluxes caused by photo-pion production in the sources of UHE cosmic rays and is a commonly used reference flux.  \FigureRef{fig:uhe-app:UHEN:UHEN-Flux-Composition-Models} shows the effects on the neutrino flux resulting from various compositions for UHE cosmic rays. This shows, as expected, a supression of the higher energy flux of neutrinos (which are comprised mainly of \Pnue, \Pnum and \APnum from photo-pion production) and associated increase in the lower energy flux (comprised of \APnue from neutron decay) as the comsic ray composition shifts to heavier nuclei.



\begin{figure}[htpb]
  \centering
  \includegraphics[width=\textwidth]{chapters/uhe-app/BZ_Neutrino_Flux.pdf}
  \caption{Predicted fluxes of \Pnue (top) and \Pnum (bottom) neutrinos from \cite{PhysRevD.64.093010}. Dashed lines correspond to neutrino fluxes, dotted lines to anti-neutrino fluxes and the sum total by solid lines.}
  \label{fig:uhe-app:UHEN:UHEN-Flux}
\end{figure}

\begin{figure}[htpb]
  \centering
  \includegraphics[width=\textwidth]{chapters/uhe-app/UHEN-Flux-Composition.pdf}
  \caption{Predictions of BZ neutrino fluxes ($\Pnu + \APnu$) for protons (black, solid), $^{4}\mbox{He}$ (green, dashed), $^{16} \mbox{O}$ (red, dash-dotted) and $^{56}\mbox{Fe}$ (blue, dotted) from \cite{Hooper200511}.}
  \label{fig:uhe-app:UHEN:UHEN-Flux-Composition-Models}
\end{figure}

\subsection{Particle physics with UHE neutrinos}
\label{sec:uhe-app:UHEN:particle-physics}

UHE neutrinos are particularly interesting from a particle physics perspective due to their enormous energies and vast distances over which they propagate. Neutrinos impacting on stationary protons can give rise to centre of mass energies an order of magnitude larger than those available at the LHC. For example a $10^{19}$ \eV neutrino would give rise to $\sim 140$ \TeV being available to produce new particles. Studying their interactions and possible divergences from those expected would be a powerful probe for new physics beyond the standard model in a regime that it is difficult to imagine replicating through current accelerator technology.

If experiments were able to distinguish between neutrino flavours there is the possibility of observing neutrino oscillations on baselines and energy scales previously inaccessible. With the current understanding of neutrino oscillations the flavour composition described in \SectionRef{sec:uhe-app:UHEN} would be maximally mixed leading to a ratio \Pnue : \Pnum : \Pnut close to 1 : 1 : 1.

Having established the observational and theoretical motivation for detecting UHE neutrinos the practicalities are now discussed.

\subsection{UHE neutrino detection}
\label{sec:uhe-app:UHEN:detection}

As neutrinos only interact via the weak force it is not possible to detect them directly, they must interact via processes shown in \FigureRef{fig:particle-physics:neutrino-interactions} and observations made of the by-products. At $10^{19}$ \eV the neutrino nucleon interaction cross-section is expected to be in the region of $0.3 \times 10^{-31} \cm^{2}$  \cite{Gandhi199681} making interactions probable in $\cubic\km$ detectors. This requires experiments to have large active volumes, too large to be purpose built, and hence naturally occurring bodies are utilised.

Cosmic ray air shower experiments, such as Auger, make use of the Earth's atmosphere as an interaction volume for cosmic rays. They also have sensitivity to neutrino interactions as the volume they see is so large. Distinctions can be made between cosmic ray  and neutrino induced air showers, although so far none of the latter have been observed. As cosmic rays (be they protons or heavy nuclei) have much higher interaction cross-sections than neutrinos the corresponding interaction length in the atmosphere is relatively short. By looking for air showers at large zenith angles (i.e. close to the horizontal) the length of atmosphere traversed by any particle interacting close enough to be detected is very large, and therefore highly unlikely to be a cosmic ray. Auger is most sensitive to \Pnut by looking for showers induced by the decay products of a \Ptau after the propagation and interaction of a \Pnut in the Earth. The inferred shower direction is required to be Earth-skimming or coming from a nearby mountain range, with the effective volume set by the decay length of \Ptau at high energies ($\sim 10 \kilo\meter$).


Many experiments rely on detecting neutrinos through the light emitted by their interaction products. In, for example, a charged current interaction between a neutrino and nucleon energy is transferred from the neutrino to a charged lepton, and nucleus, which quickly develops into a shower of charged particles. Due to the high energies involved these particles will be travelling at super-luminal speeds and Cherenkov light is emitted. Experiments such as IceCube \cite{2010RScI...81h1101H}, ANTARES \cite{2011PhLB..696...16A} and the future experiment km3Net \cite{deJong:2010zza} make use of naturally occurring materials to produce enormous detectors large enough to be sensitive to the low neutrino fluxes. IceCube, which was built around the previous AMANDA \cite{Andres:1999hm} experiment close to the geographic South Pole, uses ice as the interaction medium and observes Cherenkov light using photo-multiplier tubes (PMTs). For ANTARES and km3Net the interaction medium is water in the Mediterranean.

IceCube primarily searches for lower energy neutrinos in the \TeV-\PeV range. Cosmic ray induced \Pmu are a significant background at the lower end of this range. At these energies searches focus on up-coming neutrinos that have traversed the Earth before entering the detector from below. At high energies a different approach is taken, the outer layer of PMTs is used as a veto allowing for detection of neutrinos incident from the sides and top of the detector. These methods have proved useful in observing neutrinos over a wide range of energies, including the first observation of neutrinos believed to be of extra galactic origin \cite{PhysRevLett.111.021103} \cite{Aartsen:2013jdh} \cite{Aartsen:2014gkd}.

Despite the success of Cherenkov light based experiments at detecting energetic neutrinos, the attenuation lengths of optical light in naturally occurring media are such that the technology is prohibitively expensive to scale up to the $\sim 100 \kilo \meter ^3$ necessary to observe UHE neutrinos above \PeV.

Neutrino interactions in dense media can cause the deposit of thermal energy, via ionisation losses, which could be measured in acoustic detectors. Given sufficiently large attenuation lengths in the chosen medium this method could be used to instrument $\kilo \meter ^{3}$ scale detectors. To date only proof of principle experiments have been constructed and operated. This method has been used to instrument both ice and water as the interaction and signal propagation media through ACoRNE \cite{1742-6596-81-1-012011}, AMADEUS \cite{Lahmann2009S158}, Lake Baikal \cite{2009arXiv0910.0678A}, SAUND \cite{Kurahashi:2010ei} and SPATS \cite{Boeser:2008bj} experiments.



\subsection{The Askaryan effect and radio detection}
\label{sec:uhe-app:UHEN:Askaryan}

In 1962 Gurgen Askaryan proposed that extremely energetic particle cascades, such as those induced by neutrino interactions, in dense dielectrics could produce coherent radio pulses \cite{Askaryan1965} \cite{Askaryan1962}. Secondary electrons, positrons and gamma rays cause an electromagnetic shower to develop in the medium. Although the incident neutrino carries no electric charge a net excess in charge builds due to the presence of electrons in the medium. A combination of scattering effects cause electrons to be promoted from the medium into the shower, at the same time positrons annihilate with electrons in the medium, resulting in a net negative charge excess of $\sim 20\%$. The charge excess travels at super-luminal speeds through the medium causing Cherenkov emission which adds coherently for wavelengths greater than the shower dimensions. The power emitted scales with the number of particles in the shower squared $N^{2}$ for coherent emission, which occurs at frequencies below $1\giga \hertz$ in ice. 

This effect was experimentally confirmed in a series of experiments using the SLAC beam and a range of naturally occurring materials, initially in sand \cite{PhysRevLett.86.2802} and later with salt \cite{PhysRevD.72.023002} and ice \cite{PhysRevLett.99.171101}. In the absence of a controlled source of UHE neutrinos this was achieved by instigating electromagnetic showers using the SLAC beam. Short time duration pulses of \GeV photons were fired into a sand target in the first experiment, and in the latter two experiments beams of electrons into salt and ice, resulting in the development of electromagnetic showers over a number of meters. The expected radio emission was measured using radio antennas and exhibited the characteristic broadband frequency content, linear polarisation and coherence, as well as the scaling of emitted power with respect to number of charged particles  shown in \FigureRef{fig:uhe-app:UHEN:Askaryan:Askaryan_in_ice}. \FigureRef{fig:uhe-app:UHEN:Askaryan:Askaryan_in_salt_shape} shows an example Askaryan pulse from the SLAC beam tests \cite{PhysRevD.72.023002} which is consistent with the predicted bimodal signal with a rise time $\sim 100 \pico \second$. 


\begin{figure}[htpb]
  \centering
  \includegraphics[width=\largefigwidth]{chapters/uhe-app/askaryan_in_ice_field_strength.pdf}
  \caption{Measurements of Askaryan radiation in ice from \cite{PhysRevLett.99.171101}. The measured field strength of an Askaryan pulse as a function of frequency is shown on the left as measured by a series of different antenna designs (triangles and squares correspond to horn antennas at the top and bottom of the ANITA instrument \cite{PhysRevLett.103.051103}, whereas stars and circles are from measurements using additional antennas following discone and bicone designs). The observed power as a function of shower energy is shown on the right, demonstrating the expected quadratic dependency.}
  \label{fig:uhe-app:UHEN:Askaryan:Askaryan_in_ice}
\end{figure}

\begin{figure}[htpb]
  \centering
  \includegraphics[width=\largefigwidth]{chapters/uhe-app/askaryan_in_salt_pulse_shape.pdf}
  \caption{Askaryan pulse field strength as measured in salt from \cite{PhysRevD.72.023002}.}
  \label{fig:uhe-app:UHEN:Askaryan:Askaryan_in_salt_shape}
\end{figure}

The dimensions of such showers are determined by the properties of the medium in which they develop. The Moliere radius defines the transverse size of the shower, and hence the charge excess, and is given by:

\begin{equation}
  R_{M} = X_{0}\times21\MeV/E_{c}
  \label{eq:uhe-app:UHEN:Askaryan:Moliere}
\end{equation}

\noindent where $E_{c}$ is the critical energy and $X_{0}$ is the radiation length, both of which depend upon the medium. The radiation length is the distance over which a particle will lose all but $\frac{1}{e}$ of its energy. The critical energy is that where the ionisation loss rate is equal to that due to bremsstrahlung. Ionisation does not produce sufficiently energetic particles to add to the shower, and hence is essentially an energy loss mechanism. Bremsstrahlung, on the other hand, produces photons that can pair produce or promote atomic electrons into the shower, and is a mechanism by which the shower grows. In the case of ice the the radiation length $X_{0}\sim40\centi\meter$ and the critical energy $E_{c}\sim54\MeV$, leading to $R_{M} \sim 10\centi\meter$. The shower development occurs over several meters in ice, with $\sim90\%$ of the charge excess contained within a cylinder $R_{M}$ wide and $\sim 1\centi \meter$ long. This `pancake' of charge gives rise to coherent emission below $1 \giga \hertz$.

Since the attenuation length of radio signals in ice is an order of magnitude longer than for optical light ($\sim 1\km$ \cite{Barwick:2005-03-01T00:00:00:0022-1430:231} \cite{2004JGlac..50..522K} versus $\sim 100 \meter$ for optical), detecting UHE neutrinos via Askaryan radiation has the potential of allowing detector volumes large enough to observe BZ neutrinos. The abundance of ice in Antarctica has led to a number of pioneering efforts to detect neutrinos in this manner.

The ANITA \cite{PhysRevLett.103.051103} \cite{PhysRevD.82.022004} experiment consisted of radio antennas mounted an a long duration high altitude balloon that can see millions of $\cubic \km$ of ice. To date two flights have been made, each lasting $\sim 30$ days, with a third due for the 2014-2015 austral summer. Due to the large volume of ice that ANITA can observe the experiment is able to place  world best limits on the high energy tail of the expected GZK flux. However, the distance from neutrino interaction point leads to low signal to noise ratios, and detailed analysis is needed to remove signals of an anthroprogenic nature. The live time of such experiments is limited as there is only a short window in the summer in which it is possible to launch and recover the experiment. This also coincides with the peak of human activity on the continent and the associated increased radio background this brings with it.

The Radio Ice Cherenkov Experiment (RICE) \cite{Kravchenko200315} was composed of 18 radio antennas buried in the ice close to the south pole. The antennas operated in a frequency range of $100 \mega \hertz - 1 \giga \hertz$ and were deployed in a $200 \meter$ wide cuboid $600 \meter$ above the AMANDA neutrino telescope (which formed the precursor to the IceCube experiment). Installation of the radio antennas brought them closer to potential neutrino signals than ANITA, increasing the signal to noise ratios, however with this comes a decrease in detector volume associated with the geometry and attenuation of signals in the ice.

Two experiments currently under construction aim to have much greater sensitivity to neutrino fluxes in the $10^{17}\eV - 10^{20}\eV$ range. The Askaryan Radio Array (ARA), which will be described in detail in \ChapterRef{chap:ara-detector}, and the Antarctic Ross Ice-shelf ANtenna Neutrino Array (ARIANNA) \cite{2013ITNS...60..637K} will both consist of a large number of radio antennas buried in ice. ARIANNA will be formed of over 900 independently operating stations each of which contains 8 antennas buried in the Ross Ice Shelf, Antarctica. Neutrino induced cascades are detected via radio emission that arrives at the antennas either directly from the shower or indirectly, having reflected off the ice-sea boundary below the ice shelf.




















