\chapter{Ultra-High Energy Astro-Particle Physics}
\label{chap:uhe-app}

Since the days of Victor Hess and his spectrograph experiments much has been learned about the flux and energy of cosmic rays\footnote{In this document the term ``cosmic ray'' will refer to protons or atomic nucleii, not to gamma rays or neutrinos} incident on the Earth and many experiments have observed high energy particles of astrophysical origin. By some mechanism distant sources are able to accelerate charged particles to energies in excess of \EeV ($10^{19}$\eV), as cosmic rays with these energies have been observed for half a century. It is still unkown what the sources are or the mechanisms that produce cosmic rays at these energies, which are far in excess of those accessible by terrestrial accelerators (the current highest energy particle accelerator is the \LHC in \CERN which is able to accelerate protons to an energy of 7 \TeV). It is entirely possible that the sources of UHE cosmic rays also produce other UHE particles, observations of which will better inform the current understanding of these sources, their distribution and the acceleration mechanisms.

Currently astronomy relies on a small set of astrophysical messengers to carry information about the distant and high energy Universe, but these messengers have their limitations. UHE gamma-rays 
will pair produce \Pelectron\Ppositron off the cosmic microwave background (CMB) preventing them from travelling large distances. Unbound neutrons are unstable with a lifetime of around 15 minutes, meaning that they will decay in flight producing cosmic rays. Although CMB photons have very low energies the center of mass energy available when struck by UHE cosmic rays can be sufficient to cause photo-pion production, thus restricting the range of UHE cosmic rays. Neutrinos, on the other hand, do not suffer these horizon effects. Even at ultra-high energies weakly interacting neutrinos have such small cross-sections that they travel effectively unimpeded throughout the Universe.

Neutrinos have to this date have seen relatively little use in astronomy. For the same reasons that they are able to travel astrophysical distances they are relatively difficult to detect, as the chances of a neutrino interacting within a detector volume (and hence be observed) are relatively small. However, much has been learned from these ethereal particles. 8


%REFERENCE -- IceCube bert and ernie






\section{Cosmic rays}

\section{Ultra-high energy neutrinos}
