\chapter{Ultra-High Energy Astro-Particle Physics}
\label{chap:uhe-app}

Since the days of Victor Hess and his spectrograph experiments much has been learned about the flux and energy of cosmic rays\footnote{In this document the term ``cosmic ray'' will refer to protons or atomic nucleii, not to gamma rays or neutrinos} incident on the Earth and many experiments have observed high energy particles of astrophysical origin. By some mechanism distant sources are able to accelerate charged particles to energies in excess of \EeV ($10^{19}$\eV), as cosmic rays with these energies have been observed for half a century. It is still unkown what the sources are or the mechanisms that produce cosmic rays at these energies, which are far in excess of those accessible by terrestrial accelerators (the current highest energy particle accelerator is the \LHC in \CERN which is able to accelerate protons to an energy of 7 \TeV). It is entirely possible that the sources of UHE cosmic rays also produce other UHE particles, observations of which will better inform the current understanding of these sources, their distribution and the acceleration mechanisms.



Currently astronomy relies on a small set of astrophysical messengers to carry information about the distant and high energy Universe, but these messengers have their limitations. UHE gamma-rays 
will pair produce \Pelectron\Ppositron off the cosmic microwave background (CMB) preventing them from travelling large distances. Unbound neutrons are unstable with a lifetime of around 15 minutes, meaning that they will decay in flight producing cosmic rays. Although CMB photons have very low energies the center of mass energy available when struck by UHE cosmic rays can be sufficient to cause photo-pion production, thus restricting the range of UHE cosmic rays. Neutrinos, on the other hand, do not suffer these horizon effects. Even at ultra-high energies weakly interacting neutrinos have such small cross-sections that they travel effectively unimpeded throughout the Universe. 

Further compounding the problem of using UHE cosmic rays for astroparticle physics is that strong magnetic fields encountered in transit will bend all but the very highest energy charged particles. Although correlations between the very highest energy cosmic ray arrival directions and possible sources have been attempted, for example with the Auger data, statistics limit the significants of these correlations and to date there is no strong preference to directions associated with possible sources. Neutrinos do not suffer these bending problems meaning that they should point back to their origin. Significantly increasing the sample of cosmic rays at these energies will require some combination of new detection techniques and greatly increasing detector volumes. Another possibility is to complement these measurements with those of UHE neutrinos. As neutrinos are expected to be produced in the acceleration processes that produce cosmic rays and from interactions in transit observations of the flux and arrival directions will provide a powerful new tool with which to probe these mysteries. 



\section{Cosmic Rays}
\label{section:uhe-app:Cosmic-Rays}

Measurements of the flux of cosmic rays incident upon the Earth's atmosphere have been observed up to energies of around $10^{20}\eV$. The spectrum is steeply falling and well approximated with a power law form $dN/dE \propto E^{-\gamma}$ where $\gamma$. There are a number of features in the spectrum that are thought to be due to transitions between different classes of source and acceleration mechanisms. The three main features are: the cosmic ray \textit{knee} around $10^{15.5}\eV$, the \textit{ankle} at $3\times10^{18}\eV$, and the \textit{cut-off} above $3\times10^{19}\eV$. The spectral index below the knee is $\gamma=2.7$, steepening to $\gamma=3$ between the knee and ankle, at which it returns to a similar index to below the knee. The flux falls off rapidly such that direct detection above $10^{15}\eV$ is very difficult. At these energies the flux drops below tens of particles per m$^{2}$ year$^{-1}$ and with typical direct detection experiments using ballons or satellites, being $\sim$m$^{2}$ in size, the collection of a large enough sample is almost impossible. At these energies and above particles can only be observed in arrays that sample the extensive air showers produced by cosmic ray interactions in the atmosphere. Calculating the primary energy of the cosmic ray in these experiments is challenging as the showers involve an enormous number of particles and interactions leading to sizeable uncertainties.

The current thinking in the field is that cosmic rays at and below the energy of the knee are produced in galactic astrophysical sources. The two most popular candidate sources are supernova remnants and binary systems. It is thought that the feature described as the knee is the result of these sources reaching their maximum acceleration energy. Cosmic rays with energies above the ankle are believed to be of an extragalactic origin. There are a number of powerful astrophysical systems that may be able to accelerate particles to these energies including active galactic nucei (AGN), radio galaxies and gamma-ray bursts (GRB).


\subsection{Cosmic ray acceleration mechanisms}







