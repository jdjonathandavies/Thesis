\chapter{Particle Physics}
\label{chap:particle-physics}

\section{The standard model of particle physics}

The Standard Model of particle physics was created to describe the observed interactions of fundamental particles and has been incredibly successful in this aim. It is, however, incomplete. For one it makes no attempt to include gravity. There are also a number of observed phenomena that the model is unable to explain. One such phenomenum is neutrino oscillations and the associated non-zero masses of the neutrinos. 

The Standard Model is a  \SUgroup{2} \CrossProduct \Ugroup{1} gauge theory consisting of fermions (quarks and leptons) and fundamental forces (electromagnetic, weak nuclear and strong nuclear) which are mediated by force carrying bosons.

\subsection{The quarks}

\begin{equation}
  \begin{pmatrix}
    u \\
    d
  \end{pmatrix}
  ,
  \begin{pmatrix}
    c \\
    s
  \end{pmatrix}
  ,
  \begin{pmatrix}
    t \\
    b
  \end{pmatrix}
\end{equation}

\subsection{The leptons}

\begin{equation}
  \begin{pmatrix}
    e \\
    \nu_{e}
  \end{pmatrix}
  ,
  \begin{pmatrix}
    \mu \\
    \nu_{\mu}
  \end{pmatrix}
  ,
  \begin{pmatrix}
    \tau \\
    \nu_{\tau}
  \end{pmatrix}
\end{equation}


\subsection{The bosons}

\section{The neutrino}

