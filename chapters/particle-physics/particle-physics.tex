\chapter{Particle Physics}
\label{chap:particle-physics}

\section{The Standard Model of particle physics}
\label{section:particle-physics:SM}
The Standard Model of particle physics describes the observed interactions of fundamental particles and has proved incredibly successful. It is, however, incomplete. For one it makes no attempt to include gravity. There are also a number of observed phenomena that the model is unable to explain. One such phenomenon is neutrino oscillations and the associated non-zero masses of the neutrinos. 

The Standard Model is a  \SUgroup{2} \CrossProduct \Ugroup{1} gauge theory consisting of fermions (quarks and leptons), which are the constituents of matter, and fundamental forces (electromagnetic, weak nuclear and strong nuclear) which are mediated by force carrying bosons.

The fermions, which have spin +1/2, are grouped into generations which exhibit similar physical properties. Each fermion also has an associated anti-particle with the same mass but opposite charge. 

\subsection{The quarks}
\label{section:particle-physics:SM:quarks}

There are 6 quarks grouped into three generations. The quarks each carry fractional electric charge. The up (\Pup), charm (\Pcharm) and top (\Ptop) quarks carry +2/3 whilst the down (\Pdown), strange (\Pstrange) and bottom (\Pbottom) quarks carry -1/3 the charge of the electron.

They are arranged as follows:

\begin{equation}
  \begin{pmatrix}
    u \\
    d
  \end{pmatrix}
  ,
  \begin{pmatrix}
    c \\
    s
  \end{pmatrix}
  ,
  \begin{pmatrix}
    t \\
    b
  \end{pmatrix}
.
\end{equation}

In addition to electric charge, the quarks carry both weak and colour charge. Each quark can take on one of three colour charges named red, green and blue (and for anti-quarks anti-red, anti-green and anti-blue). Quarks are not found in isolation due to colour confinement, a property of quantum chromodynamics. When two coloured particles are moved apart, for example a quark and anit-quark (\Pq \Paq), the strong force between them increases linearly with separation. The energy stored in this field will increase to the point where another \Pq \Paq pair can be formed. The net result is two pairs of \Pq \Paq, each of which is colourless. Instead of existing in isolation quarks combine to form hadrons, colourless objects consisting of two or three quarks. Hadrons with three quarks are known as baryons, typical examples are the proton (\Pup\Pup\Pdown) and the neutron (\Pup\Pdown\Pdown). Two quark hadrons are known as mesons, consisting of a quark and anti-quark, a typical example being the  \Ppiplus (\Pup\APdown).

\subsection{The leptons}
\label{section:particle-physics:SM:leptons}

The leptons are a family of fermions, similar to the quarks, which can carry electric and weak charge. They are grouped into three generations analogous to the quarks:


\begin{equation}
  \begin{pmatrix}
    e^{-} \\
    \Pnue
  \end{pmatrix}
  ,
  \begin{pmatrix}
    \mu^{-} \\
    \Pnum
  \end{pmatrix}
  ,
  \begin{pmatrix}
    \tau^{-} \\
    \Pnut
  \end{pmatrix}
.
\end{equation}

\noindent
The electron, muon and tau all carry an electric charge of -1. Each of the charged leptons has an associated neutrino, a particle with no electric charge. In the Standard Model neutrinos have no mass and therefore travel at the speed of light.


\subsection{The bosons}
\label{section:particle-physics:SM:bosons}

In the Standard Model the interactions of fundamental particles are described through the exchange of particles with integer spin called bosons. Properties of the bosons influence the macroscopic characteristics of the forces. For example the mass of the bosons influences the effective range of the forces.

The Standard Model describes three of the four fundamental forces. Gravity is not described in the Standard Model due in part to its relatively small strength, making its effects negligible compared with other fundamental particle interactions.

The electromagnetic force is mediated through the exchange of a massless spin 1 particle called the photon, \Pphoton. The photon couples to particles with non-zero electric charge hence all fermions, other than neutrinos, interact electromagnetically. As the photon has no mass the effective range of the electromagnetic force is infinite. The weak nuclear force is mediated through the charged \PWplus, \PWminus and chargeless \PZzero bosons, which couple to weak isospin. All fermions have non-zero weak isospin and so they all interact via the weak force. The \PWpm have masses of 80.45 \GeV and the \PZzero has a mass of 91.2 \GeV. Although the weak and electromagnetic forces have similar strength the relatively large mass of these bosons makes the force appear weak and short ranged. The strong nuclear force is mediated via 8 massless gluons, \Pgluon, which couple to particles with colour charge. Since quarks are the only fermions with colour charge they are the only constituents of matter that feel the strong force. Although gluons are massless the properties of quantum chromodynamics mean that the strength of the force increases with separation, leading to the absence of isolated quarks (or coloured objects) in nature.

%REFERENCE -- W+- Z0 masses


\section{The neutrino}
\label{section:particle-physics:neutrino}

As neutrinos are electrically neutral and possess no colour charge they only interact via the weak force, making their detection more difficult than electrically charged particles such as the electron. The existence of the neutrino was first postulated in 1930 by Wolfgang Pauli, who named it the \textit{neutron}, to maintain the conservation of energy and momentum in beta decays. In 1934 Enrico Fermi coined the name \textit{neutrino}, meaning little neutral one, after the discovery by James Chadwick of a heavier neutral particle known to this day as the neutron.

The absence of electrical and colour charge for neutrinos means that they are only able to interact via the exchange of the \PWpm and \PZzero bosons. Example Feynman diagrams of charged current (CC), involving the exchange of \PWpm, and neutral current (NC), involving the exchange of \PZzero, neutrino interactions are shown in \FigureRef{fig:particle-physics:neutrino-interactions}. By observing charged leptons or hadronic recoils produced in these interactions it is possible to detect neutrinos and study their properties.

\begin{figure}[htpb]
  \parbox{100mm}{
    \unitlength=1mm
    \vspace*{5mm}
    \subfloat[CC $\nu$-$e^-$ Scattering]{
      \begin{fmffile}{CC-Electron}
	\fmfframe(0,0)(0,5){
	  \begin{fmfgraph*}(35,40)
	    \fmfstraight
            \fmfleft{i1,i2}
            \fmfright{o1,o2}
            \fmf{fermion}{i1,v1,o1}
            \fmf{fermion}{i2,v2,o2}
            \fmf{photon,tension=1.0,label=$W^{\pm}$, label.side=left}{v1,v2}
            \fmflabel{$e^{-}$}{i1}
            \fmflabel{$\nu_\alpha$}{i2}
            \fmflabel{$\nu_e$}{o1}
            \fmflabel{$l^-_\alpha$}{o2}
	  \end{fmfgraph*}
	}
      \end{fmffile}
      \label{fig:particle-physics:CC-Electron}
    }
    \hfill
    \subfloat[NC $\nu$-$e^-$ Scattering]{
      \begin{fmffile}{NC-Electron}
	\fmfframe(0,0)(0,5){
	  \begin{fmfgraph*}(35,40)
	    \fmfstraight
            \fmfleft{i1,i2}
            \fmfright{o1,o2}
            \fmf{fermion}{i1,v1,o1}
            \fmf{fermion}{i2,v2,o2}
            \fmf{photon,tension=1.0,label=$Z^0$, label.side=left}{v1,v2}
            \fmflabel{$e^{-}$}{i1}
            \fmflabel{$\nu_\alpha$}{i2}
            \fmflabel{$e^{-}$}{o1}
            \fmflabel{$\nu_\alpha$}{o2}
	  \end{fmfgraph*}
	}
      \end{fmffile}
      \label{fig:particle-physics:NC-Electron}
    }
    \vspace*{8mm}
    \\
    \subfloat[CC $\nu$-$N$ Scattering]{
      \begin{fmffile}{CC-Nucleon}
	\fmfframe(0,0)(0,5){
          \begin{fmfgraph*}(35,40)
  	    \fmfstraight
            \fmfleft{i1,i2}
            \fmfright{o1,o2}
            \fmf{fermion}{i1,v1}
            \fmf{fermion}{v1,o1}
            \fmf{fermion}{i2,v2,o2}
            \fmf{photon,tension=1.0,label=$W^{\pm}$, label.side=left}{v1,v2}
            \fmflabel{$N$}{i1}
            \fmflabel{$\nu_\alpha$}{i2}
            \fmflabel{$Shower$}{o1}
            \fmflabel{$l^-_\alpha$}{o2}
            \fmfblob{.20w}{v1}
            \fmffreeze
            \fmfi{plain}{vpath (__i1,__v1) shifted (thick*(1,-2))}
            \fmfi{plain}{vpath (__v1,__o1) shifted (thick*(-1,-2))}
            \fmfi{plain}{vpath (__i1,__v1) shifted (thick*(-1,2))}
            \fmfi{plain}{vpath (__v1,__o1) shifted (thick*(1,2))}
          \end{fmfgraph*}
	}
      \end{fmffile}
      \label{fig:particle-physics:CC-Nucleon}
    }
    \hfill
    \subfloat[NC $\nu$-$N$ Scattering]{
      \begin{fmffile}{NC-Nucleon}
	\fmfframe(0,0)(0,5){
          \begin{fmfgraph*}(35,40)
  	    \fmfstraight
            \fmfleft{i1,i2}
            \fmfright{o1,o2}
            \fmf{fermion}{i1,v1}
            \fmf{fermion}{v1,o1}
            \fmf{fermion}{i2,v2,o2}
            \fmf{photon,tension=1.0,label=$Z^0$, label.side=left}{v1,v2}
            \fmflabel{$N$}{i1}
            \fmflabel{$\nu_\alpha$}{i2}
            \fmflabel{$Shower$}{o1}
            \fmflabel{$\nu_\alpha$}{o2}
            \fmfblob{.20w}{v1}
            \fmffreeze
            \fmfi{plain}{vpath (__i1,__v1) shifted (thick*(1,-2))}
            \fmfi{plain}{vpath (__v1,__o1) shifted (thick*(-1,-2))}
            \fmfi{plain}{vpath (__i1,__v1) shifted (thick*(-1,2))}
            \fmfi{plain}{vpath (__v1,__o1) shifted (thick*(1,2))}
          \end{fmfgraph*}
	}
      \end{fmffile}
      \label{fig:particle-physics:NC-Nucleon}
    }
  }
  \caption[Charged current (CC) and neutral current (NC) neutrino interactions]{Two charged current (CC) and two neutral current (NC) neutrino interactions via which it is possible to detect neutrinos.}
  \label{fig:particle-physics:neutrino-interactions}
\end{figure}

In the Standard Model the weak force only couples to left (right) handed (anti-fermions) fermions, meaning that the neutrino, which has no electric or colour charge, can only exist as a left (right) handed particle (anti-particle). This also implies that neutrinos must be massless in the Standard Model as an additional right handed neutrino would be required to generate mass.

Measurements of neutrino interactions and properties are challenging as these particles only interact via the weak force. Many of these properties are only just becoming accessible to experimental physicists. Observations of neutrino oscillations and non-zero neutrino mass have been made in recent years, both of which are inconsistent with the Standard Model description of neutrinos as massless particles.


\subsection{Neutrino oscillations}
\label{section:particle-physics:neutrino:oscillations}

The first experimental evidence for the phenomenon of neutrino oscillations came from the Homestake Experiment in 1968 \cite{0004-637X-496-1-505}. The experiment made the first observation of neutrinos from the sun via CC interactions, but observed a deficit of solar \Pnue compared to theoretical predictions based upon solar models. One possible solution to this problem was provided by neutrino oscillations, in which neutrinos oscillate between flavour states in transit from the sun. A \Pnue produced in the sun can oscillate to a \Pnum or \Pnut in transit, both of which could not be observed in the Homestake Experiment since they have energy below the threshold to produce a muon or tau.

Subsequent experiments such as the Sudbury Neutrino Observatory (SNO) \cite{Ahmad:2002jz} and Super Kamiokande  \cite{PhysRevLett.81.1562} provided measurements of the solar and atmospheric neutrino fluxes. In 1998 Super Kamiokande provided the first experimental evidence for atmospheric neutrino oscillations by making observations of the zenith angle dependance of their observed \Pnum and \Pnue flux. Observations of neutrino CC and NC interactions (the latter being sensitive to \Pnue, \Pnum and \Pnut) in SNO also provided compelling evidence for the oscillation hypothesis and measurements of the so called `solar' neutrino oscillation parameters.

The current theoretical understanding of neutrino oscillations is that they are caused by the three known neutrino flavour states being a superposition of three mass states \textit{$m_{1}$}, \textit{$m_{2}$} and \textit{$m_{3}$}. The relationship between the weak and mass eigenstates is given in \EquationRef{eq:particle-physics:neutrino-flavour-mass-relation}.

\begin{gather}
  |\nu_{\alpha}\rangle = \sum\limits_{i=1,2,3}U_{\alpha i}|\nu_{i}\rangle
  \label{eq:particle-physics:neutrino-flavour-mass-relation}
\end{gather}


\textit{U} is known as the Pontecorvo, Maki, Nakagawa and Sakata (PMNS) matrix \cite{1968JETP...26..984P} \cite{1962PThPh..28..870M}, where the matrix element $U_{\alpha i}$ gives the relative amplitude of mass eigenstate $\nu_{i}$ ($\nu_{i}=\nu_{1},\nu_{2},\nu_{3}$) found in flavour eigenstate $\nu_{\alpha}$ ($\nu_{\alpha}=\Pnue,\Pnum,\Pnut$). Neutrinos can only be produced in weak interactions (as they have no electric or colour charge) and will be in a specific flavour eigenstate which by \EquationRef{eq:particle-physics:neutrino-flavour-mass-relation} is a superposition of mass eigenstates. Each of the mass eigenstates travels with a different speed, consequently becoming out of phase some time later. A subsequent weak interaction will find the neutrino not in a single flavour state, as it was created, but a superposition. The upshot is that neutrinos can apparently oscillate between flavour states as they travel. 

The PMNS matrix is usually decomposed in the following fashion:
%REFERENCE -- PMNS matrix
\begin{equation}
  U = 
  \begin{pmatrix}
    1 & 0 & 0 \\
    0 & c_{23} & s_{23} \\
    0 & -s_{23} & c_{23} \\
  \end{pmatrix}
  \begin{pmatrix}
    c_{13} & 0 & s_{13}e^{-i\delta} \\
    0 & 1 & 0 \\
    -s_{13}e^{-i\delta} & 0 & c_{13} \\
  \end{pmatrix}
  \begin{pmatrix}
    c_{12} & s_{12} & 0 \\
    -s_{12} & c_{12} & 0 \\
    0 & 0 & 1 \\
  \end{pmatrix}
  \begin{pmatrix}
    1 & 0 & 0 \\
    0 & e^{i\alpha} & 0 \\
    0 & 0 & e^{i\beta} 
  \end{pmatrix}
  \label{eq:particle-physics:pmns_matrix}
\end{equation}

\noindent
Where $c_{ij}=cos(\theta_{ij})$, $s_{ij}=sin(\theta_{ij})$ and $\delta$ is a CP-violating phase. Factorising the PMNS matrix in this fashion groups the mixing parameters into sets that are probed by different types of experiments. 

An illustrative example to demonstrate the phenomenon of neutrino oscillation is to take a two flavour approximation. In such an approximation we have two flavour and two mass states which are related by:

\begin{equation}
  \begin{pmatrix}
    \nu_{\alpha} \\
    \nu_{\beta} \\
  \end{pmatrix}
  =
  \begin{pmatrix}
    cos(\theta) & sin(\theta) \\
    -sin(\theta) & cos(\theta) \\
  \end{pmatrix}
  \begin{pmatrix}
    \nu_{1} \\
    \nu_{2} \\
  \end{pmatrix}.
  \label{eq:particle-physics:neutrino-oscillation-two-flavour}
\end{equation}

\noindent
A neutrino is produced by a weak interaction in a state $\nu_{\alpha}$ with an energy $E$, then travels a distance $L$. At the source we have:

\begin{equation}
  |\nu_{(x=0)} \rangle = |\nu_{\alpha} \rangle = cos(\theta) |\nu_{1}\rangle + sin(\theta) |\nu_{2}\rangle
\end{equation}

\noindent
and after travelling a distance $L$:

\begin{equation}
  |\nu_{(x=L)} \rangle = |\nu_{\alpha} \rangle = cos(\theta) e^{ip_{1}L} |\nu_{1}\rangle + sin(\theta) e^{ip_{2}L} |\nu_{2}\rangle.
\end{equation}

\noindent
Then the probability of the neutrino being in flavour state $\nu_{\beta}$ at the distance $L$ is given by:

\begin{equation}
  \begin{split}
    P(\nu_{\alpha} \rightarrow \nu_{\beta}) & = |\langle\nu_{\beta}|\nu_{\alpha}\rangle|^{2} \\
    & = | ( -sin( \theta ) \langle \nu_{1} | + cos( \theta ) \langle \nu_{2} | )*(cos(\theta) e^{ip_{1}L} |\nu_{1}\rangle + sin(\theta) e^{ip_{2}L} |\nu_{2}\rangle)|^{2}
  \end{split}
  \intertext{and since}
  \langle \nu_{i} | \nu_{j} \rangle & = \begin{cases} 
    1,      i = j\\
    0,      i \neq j
  \end{cases}      
  \intertext{we have}
  \begin{split}
    P(\nu_{\alpha} \rightarrow \nu_{\beta}) & = | sin(\theta)cos(\theta)(-e^{ip_{1}L}+e^{ip_{2}L}) | ^{2} \\
    & = sin^{2}(2\theta)sin^2(\frac{(p_{1}-p_{2})L}{2}).
  \end{split}
  \intertext{In the limit that $E_{i} \gg m_{i}$ we arrive at:}
  \begin{split}
    P(\nu_{\alpha} \rightarrow \nu_{\beta}) & = sin^{2}(2\theta)sin^2(\frac{(1.27\Delta m_{21}^{2})L}{4E}).\\
  \end{split}
  \label{eq:particle-physics:neutrino-oscilltions:two-flavour}
\end{equation}

\noindent
Where $\Delta m_{21}^{2} = m^{2}_{2} - m^{2}_{1}$, $L$ is the distance travelled measured in km and finally $E$ is the neutrino energy in \GeV. The result is that a neutrino produced via a weak interaction, in a specific flavour state, has a non-zero probability of being measured in a different flavour state after travelling some distance. This probability oscillates as a function of distance, such that an appropriate choice of propogation length can probe the mixing parameters $\theta$ and $\Delta m_{21}^{2}$.

\subsubsection{Measurement of oscillation parameters}
\label{section:particle-physics:neutrino-oscilltions:parameters}

The initial findings of the SNO experiment have been complemented over recent years with measurements from a series of dedicated neutrino oscillation experiments. SNO made measurements of the `solar' neutrino oscillation parameters ($\theta_{12}$ and $\Delta m_{12}^{2}$) which dominate the electron neutrino survival probability at this $L/E$. Kamiokande \cite{PhysRevLett.81.2016} and it's successor Super-Kamiokande \cite{PhysRevD.71.112005} make observations of neutrinos produced in cosmic-ray induced pion and kaon decay chains to measure the `atmospheric' oscillation parameters ($\theta_{23}$ and $\Delta m_{23}^{2}$). There are two main categories of additional experiment that contribute to these measurements: accelerator and reactor experiments.

Accelerator experiments involve a neutrino beam that is measured in a near detector close to the beam's origin and then propagated over a long baseline to a far detector. This approach allows for careful selection of the energy and baseline ($E$ and $L$ in \EquationRef{eq:particle-physics:neutrino-oscilltions:two-flavour}) to gain the best measurement sensitivity. A number of experiments have used this approach to make precise measurements of the solar and atmospheric mixing parameters over the last decade \cite{PhysRevD.74.072003} \cite{PhysRevLett.101.131802} \cite{PhysRevLett.107.041801}.

 Reactor experiments typically involve a detector placed near one or many nuclear reactors and measure the flux of neutrinos produced in the reactor core. Recent results from Daya Bay among other experiments give strong evidence for non-zero $\theta_{13}$ \cite{PhysRevLett.108.171803} \cite{PhysRevLett.108.131801} \cite{PhysRevLett.108.191802}. 

The current measurements of the oscillation parameters are summarised in \TableRef{tab:particle-physics:neutrino-oscilltions:parameters}. Although there is strong experimental evidence for neutrino oscillations there are still notable gaps including the sign of $\Delta m^{2}_{23}$ and information about the CP-violating phase $\delta$.


%table -- this is the pdg 2012 table -as per james mott
\begin{table}
\begin{center}
  \begin{tabular}{ l | l }
    Parameter & Value\\
    \hline
    $\Delta m_{21}^{2}$ & $7.50 \pm 0.20 \times 10^{-5} \eV^{2}$\\
    $\lvert\Delta m^{32}\rvert$ & $2.32^{+0.12}_{-0.08} \times 10^{-3} \eV^{2}$\\
    $\sin^{2}\theta_{12}$ & $0.857 \pm 0.024$\\
    $\sin^{2}\theta_{23}$ & $>0.95$\\
    $\sin^{2}\theta_{13}$ & $0.95\pm0.010$\\
  \end{tabular}
  \caption{The best-fit values derived from a global fit to the current neutrino oscillation data \cite{Beringer:1900zz}.}
  \label{tab:particle-physics:neutrino-oscilltions:parameters}
\end{center}
\end{table}



\subsection{Neutrino mass}
\label{section:particle-physics:neutrino:mass}

The Standard Model predicts neutrinos to have zero mass, but the phenomenon of neutrino oscillations is best explained by the presence of finite non-zero mass states. Oscillation experiments are most sensitive to differences in the mass states as these directly affect the oscillations. The ordering of these masses from least to most massive is still unknown. Experiments such as NO$\nu$A \cite{Ayres:2004js} and LBNE \cite{Adams:2013qkq} will have much improved sensitivity to the mass hierarchy by observing the effects of neutrinos' passage through matter. This is achieved by having longer baselines ($L$ in \EquationRef{eq:particle-physics:neutrino-oscilltions:two-flavour}), lower energies and more intense beams than previous neutrino beam experiments.

Oscillation experiments are able to place a lower limit on the heaviest mass state. The measurement of the largest mass splitting $|\Delta_{23}^{2}|$ combined with the fact that the lightest mass cannot be less than 0 leads to a lower bound on the heaviest active mass state.


Beta decay experiments are sensitive to the neutrino mass via measurements of electron energy in $\beta$ decays. Tritium ($^3\mbox{H}$), an isotope of hydrogen, can undergo beta decay:

\begin{equation}
  ^{3}\mbox{H} \rightarrow ^{3}\mbox{He} + \Pelectron + \APnue.
  \label{eq:particle-physics:neutrino-mass:tritium-decay}
\end{equation}

\noindent The energy of the emitted electron follows a $\beta$ decay spectrum with an end point that depends upon the neutrino mass. In the case that neutrinos are massless the end point of this spectrum will be equal to the difference in rest mass energy of $^{3}\mbox{H}$ and $^{3}\mbox{He} + \Pelectron$. Since neutrinos are known to have mass the end point energy of the electron spectrum will be reduced by the neutrino mass. By making measurements of the electron energy it is therefore possible to place constraints on the absolute mass of the neutrino.

Another constraint on neutrino mass comes from cosmological measurements. The most important probe comes from anisotropies in the cosmic microwave background (CMB). A summary of these constraints is found in \TableRef{tab:particle-physics:neutrino-mass}.

\begin{table}
\begin{center}
  \begin{tabular} { l | l | l }
    Parameter & Value & Source\\
    \hline
    $m_{1}$ or $m_{3}$ &  $> 0.05 \eV$ & Oscillations\cite{Beringer:1900zz}\\
    $\sum m_{i}$ & $ < 0.28 - 0.44 \eV$ & Cosmology\cite{Hinshaw:2012aka} \cite{PhysRevLett.105.031301}\\
    $m_{\beta}$ & $ < 2.0 \eV$ & $\beta$ decay \cite{Kraus:2004zw} \cite{PhysRevD.84.112003}\\
    $\langle m_{\beta\beta} \rangle$ & $< 0.11 - 0.25 \eV$ & $0\Pnu\beta\beta$ \cite{PhysRevD.88.091301}\\
  \end{tabular}
  \caption{Experimental constraints on the neutrino mass.}
  \label{tab:particle-physics:neutrino-mass}
\end{center}
\end{table}
  






