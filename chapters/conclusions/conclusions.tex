\chapter{Conclusions}
\label{chap:Conclusions}

The analysis of data from 2011-2012 taken with the ARA prototype TestBed detector yields a limit on the UHE neutrino flux between energies of $10^{17} - 10^{21} \eV$. No neutrino candidate events were found on an expected background of $0.77^{+0.31}_{-0.43}$. Upon relaxing some of the cuts, designed to remove anthropogenic signals, a large number of events reconstructing to the Sun were observed, correlated with increased levels solar activity. Although these signals are not of an impulsive nature they demonstrate the TestBed's ability to detect sources other than calibration signals.

Work was undertaken in developing and delivering the data acquisition systems for the first full ARA stations, which were successfully deployed during 2012-2013. These new stations have much improved functionality than the TestBed and are expected to have significantly increased sensitivity to neutrino fluxes. Under reasonable assumptions the limit derived from analysis of TestBed data was extrapolated to a full 37 station array operating over a 3 year period. The projected sensitivity for the full array is an order of magnitude better than current experiments and should be able to make the first observations of GZK neutrinos when completed.
