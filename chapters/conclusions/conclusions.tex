\chapter{Conclusions}
\label{chap:Conclusions}

The analysis of data from 2011-2012 taken with the ARA prototype TestBed detector yields a model indipendent limit on the UHE neutrino diffuse flux between energies of $10^{17} - 10^{21} \eV$ shown in \FigureRef{fig:Results:Limit-Plot}. No neutrino candidate events were found on an expected background of $0.77^{+0.31}_{-0.43}$. Upon relaxing some of the cuts, designed to remove anthropogenic signals, a large number of events reconstructing to the Sun were observed, correlated with increased levels solar activity. Although these signals are not of an impulsive nature they demonstrate the TestBed's ability to detect sources other than calibration signals.

Work was undertaken in developing and delivering the data acquisition systems for the first full ARA stations, which were successfully deployed during 2012-2013. These new stations have much improved functionality than the TestBed and are expected to have significantly increased sensitivity to neutrino fluxes. Under reasonable assumptions the limit derived from analysis of TestBed data was extrapolated to a full 37 station array operating over a 3 year period. The projected sensitivity for the full array is an order of magnitude better than current experiments and should be able to make the first observations of GZK neutrinos when completed.

In the event of making an observation of GZK neutrinos with the full ARA array there are a number of improvements that would be desirable to determine both the energy and direction of propagation of neutrino primaries. Energy measurements would serve as a means to confirm that the observed events are produced by ultra-high energy neutrinos and could be used to make some determination of the energy dependent flux. Directionality of incident neutrino candidates would be of great use in ruling out possible background sources as well as being used to determine whether there are any preferential directions of origin for these particles.

A detailed calibration study would be necessary to determine the relationship between the measured RF power received in antennas and the energy of the incident neutrino. It could be possible to make measurements in both the laboratory, and the field, of the power radiated by calibration pulsers. These studies could be either linked to experiments conducted at SLAC that measured Askaryan emission \cite{PhysRevLett.99.171101} , or to dedicated beam studies using ARA equipment. Once deployed these calibration sources could be used to set the scale of received power in the receive antennas in ARA stations, which in turn can be linked to the energy contained in the particle cascade and hence incident neutrino energy. This could be achieved with the current design for calibration antennas local to each station, however with additional work before and after installation to determine the power radiated by these sources. In addition it would be of great use to have sources that are able to illuminate all ARA stations simultaneously, being used as a common, well known source with which to calibrate. The preferred solution would be to deploy high power antennas at the centre of the array at great depth such that they are able to illuminate even the farthest stations. 

In order to better determine incident neutrino directions detailed studies would be necessary to determine the received power scales between the horizontally and vertically polarised antennas. This would be needed as the ratio of power between polarisations can be used to determine which part of the Cherenkov cone was sampled in a neutrino event, and hence would be able to constrain the event topology and incident neutrino direction. In addition it would be of great use to better determine the positions of receive antennas in ARA stations, as this would lead to improvements in reconstruction of incident RF signal direction and distance from the station. Both of these improvements could be made with more calibration data from well known pulsers illuminating the ARA stations. A calibration system that would allow for movement of calibration antennas vertically within a calibration hole would be advantageous. This would allow for a large increase in the number of calibration source locations without the expense of additional drilling. This could be achieved with an electronically controlled system that raises or lowers the antenna by known distances, however this would require a large amount of effort to develop and deploy reliably.


With these suggested improvements ARA 37 would provide an excellent window through which to make the first observation of GZK neutrinos.
