\chapter{Introduction}
\label{chap:introduction}

Over a hundred years ago Victor Hess pioneering work led to the discovery of cosmic rays - charged particles bombarding the Earth's atmosphere. Hess measured that the amount of ionising radiation increases with altitude and deduced that this radiation came from outside the Earth's atmosphere, which had a shielding effect. The discovery of these particles was of fundamental importance in two ways: by observing in detail their interactions a `zoo' of seemingly more fundamental particles were discovered, which led in turn to the birth of particle physics; furthermore astronomers had a new window other than light through which to observe the Universe.

For over 50 years cosmic rays have been observed with arrival energies in excess of $10^{18}$ \eV, energies well beyond those which can be achieved in terrestrial particle accelerators. However, there is much that is still unkown about the mechanisms that produce them or their sources. At these energies both cosmic rays and photons suffer from horizon problems that limit their use as astrophysical messengers. Ultra-high energy (UHE) neutrinos do not suffer such horizon problems and their observation and study could lead to a new window into the the distant Universe and the possibility of studying particle physics at energies well beyond those produced in the laboratory. 

Unlike cosmic rays neutrinos only interact weakly and are not bent by galactic and extra-galactic magnetic fields, meaning that they will point back to their sources. Since neutrino interactions with matter are weak in comparison with other astrophysical messengers they are able to travel cosmological distances uninhibited. The horizon effects that limit the range and flux of the highest energy cosmic rays are also expected to produce a flux of UHE neutrinos whilst in transit, but also at the sources. Measurement of these neutrinos will provide an insight into some of the unkowns about cosmic ray production mechanisms and source distributions.

The age of neutrino astronomy is edging closer, but the technical challenges of detecting UHE neutrinos are considerable. The UHE cosmic ray flux decreases with energy and it is expected that the neutrino flux will mirror this. The low flux combined with the small interaction cross-sections requires enormous detector volumes.

The Askaryan Radio Array (ARA) is an experiment based at the South Pole, Antarctica designed to measure coherent Cherenkov radio emission from neutrino enduced particle cascades in the polar ice sheet. This coherent radio emission, known as Askaryan radiation, was measured in a series of experiments at SLAC in a range of dense dielectrics, including ice. Attenuation lengths of radio waves in ice are of the order of 1km, an order of magnitude greater than optical signals, make the possibility of instrumenting large detector volumes a more affordable and achievable prospect.

ARA's current design is for an array of 37 sub-detectors, or stations, each capable of functioning as a stand-alone neutrino detector. Each station consists of a series of deep holes in the ice containing radio antennas. Data is recorded and trigger decisions made by custom electronics and computer for each station. A prototype station, dubbed the TestBed, was deployed in the 2010 Austral summer and collected data for 2 years autonomously. A further 3 stations with upgraded functionality were deployed in the 2012-2013 Austral summer, with further deployments planned over the coming years until ARA reaches its design goal.

The work described in this thesis will contain an analysis of data from the TestBed 2011-2012 data set. 

%FIXME -- this will be the tag for bits that need fixing

%PLOT -- this will be the tag for where plots should be

%INSERT -- this will be the tag for where we should insert something like text

%REFERENCE -- this will be the tag for a reference
