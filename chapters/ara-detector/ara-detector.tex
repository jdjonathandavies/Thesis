\chapter{The Askaryan Radio Array}
\label{chap:ara-detector}

Radio detection of neutrinos is a promising experimental method in the search for UHE neutrinos. The likes of the ANITA \cite{PhysRevLett.103.051103} \cite{PhysRevD.82.022004} and RICE \cite{Kravchenko200315} experiments have shown that it is feasible to instrument large volumes of ice relatively inexpensively. However, to date no UHE neutrinos have been observed with such experiments and the next generation detectors must find a way of improving sensitivity by an order of magnitude or more in the $10^{17} - 10^{18}\eV$ region. The Askaryan Radio Array (ARA) is one such experiment that looks to build on the pioneering work by other experiments in the field. ARA will consist of a series of antenna clusters, or stations, buried deep in the ice near the Amundsen-Scott South Pole Station. Each of these antenna clusters will have dedicated triggering and digitisation electronics to enable them to operate as stand-alone neutrino detectors. This design lends itself to a phased installation that will allow the detector volume to grow as sensitivity is required, with the aim of first establishing an UHE neutrino flux and then giving the option to expand to an observatory class detector to make detailed measurements. 

By burying the antennas in the ice the signal to noise ratio expected from neutrino signals increases significantly compared with balloon-borne experiments, such as ANITA, leading to a decrease in the energy threshold but with an associated decrease in detector volume due to the geometry and attenuation of radio signals within the ice. In addition the stations are able to operate year round and are not limited to the relatively noisy summer season in which human activity is at its peak. The ANITA experiment has to remove significant amounts of anthroprogenic radio signals from their data set. By choosing a location that is relatively isolated, and by collecting data year round ARA will be able to greatly reduce the limitations that this places on live time and hence neutrino sensitivity.

The ice sheet upon which the South Pole Station sits is $\sim 2.8 \km$ thick and has exceptional radio clarity \cite{Barwick:2005-03-01T00:00:00:0022-1430:231} \cite{2004JGlac..50..522K}. \FigureRef{fig:ara-detector:Ice-Attenuation} shows the results of ice attenuation measurements taken at the South Pole which results in an attenuation length estimate of $L_{\alpha} = 1450^{+300}_{-150} \meter$ at a frequency of $380 \mega \hertz$ and temperature of $-50\celsius$.

\begin{figure}[htpb]
  \includegraphics[width=\largefigwidth]{chapters/ara-detector/Barwich-Ice-Attenuation.jpg}
  \caption{Ice attenuation lengths as a function of frequency from \cite{Barwick:2005-03-01T00:00:00:0022-1430:231}. The lower set of lines correspond to average attenuation lengths under various assumptions for reflectivity of the bedrock below the ice sheet. The open pentagonal symbols are obtained by normalising the transmitted and received signals in the air relative to in the ice. The upper set of lines show derived attenuation lengths taking into account the temperature profile in the ice.}
  \label{fig:ara-detector:Ice-Attenuation}
\end{figure}

The top layer of ice, known as the firn, consists of compacted snow of lower density than solid ice,  leading to a varying index of refraction through this layer. This results in ray-bending and shadowing affects that limit the effective volume of a detector deployed within it, whilst complicating triggering and reconstruction of incident radio signals' origin. The firn typically extends for $\sim 150 \meter$, below which the index of refraction changes little. There is a clear benefit to installing antennas below this layer to circumvent these effects, but the costs of drilling rise rapidly with depth. The challenge of drilling wide, deep and dry holes to these depths is not insignificant, although ARA is able to draw on the expertise within the IceCube collaboration which successfully deployed photo-multiplier tubes at depths of $1.5 - 2.5 \kilo \meter$ \cite{2010RScI...81h1101H}.


The expected signal from neutrino induced Askaryan radiation is a highly linearly-polarised radio frequency (RF) impulse. This will come in the form of a spherical wave-front, which is well approximated by a plane wave for distant sources. The antennas effectively sample this wave-front and the timing differences between signals received in pairs of antennas can be used to identify both the source direction and the distance from the station. By measuring the polarisation of detected signals it is possible that additional information can be obtained about which part of the Cherenkov cone each antenna has sampled, hence providing another observable with which to determine the event topology. The receive antennas used within a station should have a dipole response and be split into polarisation along two orthogonal directions. This allows for detection of RF signals polarised along an arbitrary direction removing any bias to a particular polarisation.

The impulsive nature of the Askaryan signal, with experimentally observed rise times of $< 100 \pico \second$ \cite{PhysRevLett.86.2802}, necessitates high sampling rate digitisation of the analogue signal received by the antennas. Power is at a premium in such remote locations, placing a severe constraint on the consumption of the trigger and data acquisition systems. Application Specific Integrated Circuits (ASICs) are used that meet these challenging requirements.



\section{ARA 37}
\label{sec:ara-detector:ARA37}

\begin{figure}[htpb]
  \centering
  \includegraphics[width=\hugefigwidth]{chapters/ara-detector/ARA-37.pdf}
  \caption{\textit{Left} the ARA-37 layout including TestBed and ARA1-3 provided by Ryan Manu. \textit{Right} schematic for idealised station similar to those deployed as ARA1-3.}
  \label{fig:ara-detector:ARA-37:ARA-37}
\end{figure}

The current proposed design for ARA is a hexagonal arrangement of 37 stations named `ARA37' shown in \FigureRef{fig:ara-detector:ARA-37:ARA-37}. The current design for ARA37 is driven by the aim of observing a flux of UHE neutrinos, and as such this phase will essentially be a counting experiment. The design choices, in terms of number of antennas per station and the geometry of stations within the detector as a whole, are optimised to maximise detector volume. This does not preclude multi-station events, which would provide extra information to determine event details and energies. These events, however, are expected to be rare in comparison with single station events. Once a UHE neutrino flux is established it would be possible to in-fill with stations to improve energy resolution and better distinguish between classes of neutrino event. Tau neutrino interactions, for example, can lead to a secondary decay of a \Ptau within the detector volume. In addition interactions of \Pnum can produce muons that undergo photonuclear energy losses spatially separated from the initial shower \cite{MercurioThesis}, also leading to Askaryan emission. Such `double-bang' events could produce two particle showers being reconstructed using separate stations.


The modular design enables the individual detectors to be developed as the array is installed, informed by the experience of installation, operation and the data from those previously deployed. A prototype station, known as the `TestBed', was installed during the austral summer 2010-2011, which is marked in \FigureRef{fig:ara-detector:ARA-37:ARA-37}, and is described in detail in \SectionRef{sec:ara-detector:TestBed}. 



\section{The TestBed}
\label{sec:ara-detector:TestBed}

During the austral summer of 2010-2011 a prototype station was deployed in the ice designed to provide a tool with which to asses the suitability of the ice, the radio environment, to aid the development of future stations and provide data with which to conduct the first neutrino analysis. The TestBed consists of 14 horizontally (HPol) and vertically (VPol) polarised \footnote{Horizontally (Vertically) polarised antennas will be referred to as HPol (VPol) in this document.} antennas deployed in the ice, alongside 2 horizontally polarised surface antennas and a dedicated digitisation, triggering and data recording unit known as the `DAQ box'. The surface antennas were intended for use in searching for radio emission produced by cosmic ray air showers. The layout of the TestBed is shown diagrammatically in \FigureRef{fig:ara-detector:TestBed:Schematic}.

\begin{figure}[htpb]
  \centering
  \includegraphics[width=\hugefigwidth]{chapters/ara-detector/TB-Schematic.pdf}
  \caption{The layout of the TestBed provided by Eugene Hong.}
  \label{fig:ara-detector:TestBed:Schematic}
\end{figure}

The dedicated hot-water drilling equipment necessary to drill below the firn layer was not yet available limiting the depth of holes to $\sim 30 \meter$. 6 HPol and 4 VPol were deployed at a depth of $30 \meter$, the rest to a depth of $\sim 1 \meter$. The specifications for various aspects of the TestBed are summarised in \TableRef{tab:ara-detector:TestBed:Specifications} and can be compared to the future stations design shown in \TableRef{tab:ara-detector:ARA1-3:Specifications}.


\begin{table}
\begin{center}
  \begin{tabular}{ p{0.48\textwidth} p{0.48\textwidth} }
    \textbf{Specified Parameter}  & \textbf{TestBed Station} \\    
    \hline

    Number of VPol antennas                            & 2 near-surface, 4 in ice       \\
    VPol antenna type                                  & bicone                         \\
    VPol antenna bandwidth (\mega\hertz)               & 150-850                        \\
    Number of HPol antennas                            & 2 near-surface, 6 in ice       \\
    HPol antenna type                                  & BSC \& QSC        \\
    HPol antenna bandwidth (\mega\hertz)               & 250-850                        \\
    Surface antenna type                               & fat dipole                     \\
    Surface antenna bandwidth (\mega\hertz)            & 30-300                         \\
    Number of surface antennas                         & 2                              \\
     Number of receive antenna boreholes                & 4                              \\
    Borehole depth (\meter)                            & 30                             \\
    Vertical antenna configuration                     & VPol (HPol) above HPol (VPol)  \\
    Vertical spacing (\meter)                          & 5                              \\
    Approximate geometry                               & trapezoidal                    \\
    Approximate radius (\meter)                        & 10                             \\
    Number of calibration antenna boreholes            & 3                              \\
    Calibration borehole distance from centre (\meter) & 30                             \\
    Calibration hole geometry                          & equilateral triangle           \\
    Calibration signal type                            & impulse only                   \\
    LNA noise figure (\kelvin)                         & $< 80$                         \\
    LNA/amplifier dynamic range                        & 30:1                           \\
    RF amplifier total gain (dB)                       & $>75$                          \\
  \end{tabular}
  \caption{TestBed detector specifications.}
  \label{tab:ara-detector:TestBed:Specifications}
\end{center}
\end{table}

The TestBed station geometry differs from the design layout for future stations due, in part, to the logistical challenges faced for the initial deployment. The deployed positions and antenna types are summarised in \TableRef{tab:ara-detector:TestBed:Positions} in a station-centric coordinate system. The coordinate system chosen has $+\hat{x}$ in the direction of ice flow, the $\hat{x} - \hat{y}$ plane tangent to the earth's geoid shape at the surface and is centred on the south east corner of the DAQ box on the surface of the ice \cite{TestBedGeom}. 

\begin{table}
\begin{center}
  \begin{tabular}{ l l c c c}
    \textbf{Hole} & \textbf{Antenna} & \multicolumn{3}{c}{\textbf{Position}}\\
    \cline{3-5}
    & & \textbf{X (\meter)} & \textbf{Y (\meter)} & \textbf{Depth (\meter)}\\
    \hline 
    BH 1 & BSC, HPol & -8.42 & -4.40 & 20.50 \\
    & Bicone, VPol &  &  & 25.50 \\
    BH 2 & BSC, HPol & -0.42 & -11.13 & 27.51 \\
    & Bicone, VPol &  &  & 22.51 \\
    BH 3 & BSC, HPol & 9.22 & -6.15 & 22.73 \\
    & Bicone, VPol &  &  & 27.73 \\
    BH 5 & BSC, HPol & 3.02 & 10.41 & 30.56 \\
    & Bicone, VPol &  &  & 25.56 \\
    BH 6 & QSC, HPol & -9.07 & 3.86 & 26.41 \\
    & QSC, HPol &  &  & 30.41 \\
    S1 & Discone, VPol & -2.48  & -1.75  & 1.21 \\
    & Batwing, HPol &  &  &  2.21\\
    S2 & Batwing, HPol  & 4.39  & -2.41 &  1.19\\
    S3 & Discone, VPol & 1.58  & 3.80  &  1.19\\
    S4 & Fat Dipole, HPol  &  &  &  \\
    Cal1 & HPol & -23.18  & 17.90  &  17.50\\
    & VPol &   &  &  22.50\\
    Cal2 & HPol & -2.25 & -29.81 &  34.23\\
    & VPol &  &  &  29.23\\
    Cal3 & HPol & 27.58 & 13.63 &  1.13\\
    & VPol &  &  &  1.13\\
  \end{tabular}
  \caption{TestBed boreholes, antenna types and deployed positions.}
  \label{tab:ara-detector:TestBed:Positions}
\end{center}
\end{table}


\subsection{Signal chain}
\label{sec:ara-detector:TestBed:Signal-Chain}

The TestBed signal chain is shown in figure \FigureRef{fig:ara-detector:TestBed:Signal-Chain}. A number of different antenna types are used both down-hole and near the surface, the signals from which are used for triggering and digitisation in the DAQ box. The main constraints on antenna design come from the need to fit into drilled boreholes and to allow for multiple antennas in each of these. The feasibility of drilling $15 \centi\meter$ wide holes to depths of $200 \meter$ has been demonstrated and the antennas must fit within a cylinder of this size. To minimise the number of holes necessary per station multiple antennas are placed in each borehole, this means that the antenna design must be such that they allow for feed-through cables from lower antennas. To prevent any directional bias in sensitivity the antennas should have approximate azimuthal symmetry in their response. 

\begin{figure}[htpb]
  \centering
  \includegraphics[width=\hugefigwidth]{chapters/ara-detector/TB-Antennas.pdf}
  \caption{ARA TestBed down-hole antennas from  \cite{Allison2012457}. The left two images are of the bicone VPol antennas, the right two images are of the bowtie-slotted-cylinder HPol antennas.}
  \label{fig:ara-detector:TestBed:Antennas}
\end{figure}

The VPol antennas used in the boreholes for the TestBed have a wire-frame hollow-centre biconical design shown in \FigureRef{fig:ara-detector:TestBed:Antennas}, where the feed region is annular around the pass-through cable. The HPol antenna design is significantly more challenging than that for the VPol antennas and two designs were implemented for testing in the TestBed: a bowtie-slotted-cylinder (BSC) antenna, and a quad-slotted-cylinder (QSC) antenna with internal ferrite loading to lower its frequency response. The design goal was to produce antennas to cover a frequency range of $150 \mega \hertz$ to $850 \mega \hertz$. This was achieved with the VPol antennas but proved difficult for both designs of HPol antennas, which struggled to obtain the required response below about $200 \mega \hertz$ to $250 \mega \hertz$ in ice as shown in \FigureRef{fig:ara-detector:TestBed:Signal-Chain:Frequency-Response}. Although the performance of the QSC HPol antennas was measured to be significantly better than the BSC antennas the latter were deployed in the boreholes due to manufacturing constraints ahead of deployment.

\begin{figure}[htpb]
  \centering
  \includegraphics[width=\textwidth]{chapters/ara-detector/Antenna-Response.pdf}
  \caption{Frequency response of (left) bicone VPol antennas and (right) the bowtie-slotted-cylinder HPol antennas from \cite{Allison2012457}. The equivalent power transmissivity as a function of frequency  (bottom left and top right) shows the expected broadband response in both polarisations. The voltage standing wave ratio (VSWR) is also shown (top left and bottom right).}
  \label{fig:ara-detector:TestBed:Signal-Chain:Frequency-Response}
\end{figure}


One of the challenges of deploying receive antennas in the ice is to get signals to the DAQ box with little signal degradation. A number of design choices are made in the signal chain with this in mind. Expected signal sizes are very small and transmission distances from antenna to DAQ box $\sim 50\meter$ for the TestBed (and $\sim 300\meter$ for future stations) so several stages of amplification are used to boost the signals before their arrival at the DAQ box. Roughly $1 \meter$ above the antennas signals are filtered and amplified before transmission to the surface. The filters define the band between $150 \mega\hertz$ and $850 \mega\hertz$ and notch out a particularly strong in band frequency at $450 \mega \hertz$ use for South Pole communications. The output of the initial filters is sent through a low noise amplifier (LNA). The placement of the filters before the first stage of amplification serves two purposes: firstly this lowers the insertion of out of band thermal noise into the signal chain, secondly it prevents strong out of band signals, for example from south pole communications ($450 \mega\hertz$), from saturating the LNA. Large signals input to the LNA can cause it to be pushed into a non-linear mode of operation, leading to compression \footnote{Compression in RF amplifiers is a phenomenon whereby linearity of amplifier gain is lost. Large signals are amplified less than small signals leading to a loss of dynamic range}.

Signals are transmitted from down-hole to the surface using coaxial RF cables. Transmission in such a manner causes some signal loss and insertion of noise, however this is not a large effect in the TestBed due to the short distances involved. For future stations in which the transmission distance can be $\sim 300 \meter$ these losses become a real challenge, leading to large decreases in the signal to noise ratio between the output of the LNAs and the surface. In \SectionRef{sec:ara-detector:ARA1-3} the modified signal chain for future ARA stations will be discussed.

\begin{figure}[htpb]
  \centering
  \includegraphics[width=\hugefigwidth]{chapters/ara-detector/TB-Signal-Chain.pdf}
  \caption{Block diagram of the TestBed signal chain from \cite{Allison2012457}.}
  \label{fig:ara-detector:TestBed:Signal-Chain}
\end{figure}

Measurements, made in the laboratory ahead of installation, of the signal chain noise temperature and total gain are shown in \FigureRef{fig:ara-detector:TestBed:Signal-Chain:Gain-Noise}. The combination of filters and amplifiers contributes approximately $80 \kelvin$ of noise, with a further $230 \kelvin$ expected from the ambient temperature of the ice seen by the receive antennas. The total system noise is expected to be $310 \kelvin$, and is dominated by the surrounding ice. This leads to an input power of -85dBm at the LNA which is amplified by 80dB by the signal chain ahead of input to the DAQ box for digitisation.

\begin{figure}[htpb]
  \centering
  \includegraphics[width=\largefigwidth]{chapters/ara-detector/TestBed-Signal-Chain-Gain-Noise.pdf}
  \caption{Total gain (top) and noise figure (bottom) for the TestBed signal chain (preamplifiers and receivers) from \cite{Allison2012457}. The notch filter at $450 \mega\hertz$ is clearly visible in both. The gain falls off at high frequencies due to the presence of a low pass filter at $850 \mega \hertz$. The two lines to the left on each figure (green and blue) are for the surface antenna signal chains.}
  \label{fig:ara-detector:TestBed:Signal-Chain:Gain-Noise}
\end{figure}

Once a signal has propagated from the antenna through to the DAQ box the signal is split into two paths: one for triggering, and the other for digitisation. These paths are described in the following sections.

\subsection{Triggering}
\label{sec:ara-detector:TestBed:Triggering}

The power and data rate restrictions from ARA's location mean that it is not possible to continuously record data from the receive antennas. As a result a trigger is implemented to select interesting events that are then recorded by the digitisation electronics for later analysis. The main backgrounds for ARA are from thermal noise and anthroprogenic signals. ARA's location was chosen to minimise the latter, which is expected to affect a reasonably small fraction of live time for the TestBed and nearly zero for future stations, which are further from noise sources and buried deeper in the ice. The trigger is therefore designed to distinguish between impulsive RF signals, such as those expected from Askaryan radiation, and random thermal fluctuations. This is done by requiring that some interesting pattern of RF signals has crossed an intensity threshold in the trigger chain within a narrow time window, at which point a trigger is asserted. This triggers the digitisation of the voltage-time waveforms from all of the receive antennas in a window around the time the trigger was asserted, which is referred to as an event.

In the triggering chain of each antenna a coaxial tunnel diode is used to provide a uni-polar signal that is proportional to the RF power integrated over a few nano seconds. These signals are then fed into a discriminator which determines whether the RF power has crossed an an adjustable threshold. The discriminator is implemented via a field programmable gate array (FPGA) using low-voltage differential signal comparators. For each antenna that passes the defined threshold a one-shot \footnote{the one-shot corresponds to a `trigger passed' signal being asserted for a limited period of time} is generated in the FPGA, these signals can then in turn be assessed by firmware logic to determine if a trigger condition has been met. In the TestBed the trigger condition is that 3 of the 8 VPol and HPol borehole antennas produce signals that pass threshold within a $100 \nano\second$ window. 

The TestBed coincidence trigger leads to a dependence on the individual threshold-crossing rates, or singles rates $R_{single}$ of each antenna. As thermal noise is essentially random the global trigger rate on thermal signals is given to first order by:


\begin{equation}
  R_{global} = N C^{N}_{M}R_{single}^{N}t^{N-1}
  \label{eq:ara-detector:TestBed:Trigger-Rate}
\end{equation}

\noindent for $N$ of $M$ coincidence in a time window $t$, where in the TestBed $N=3$, $M=8$ and $t=100\nano\second$. The trigger efficiency was measured in the laboratory as a function of signal to noise ratio (SNR) for an input impulsive signal and is shown in \FigureRef{fig:ara-detector:TestBed:Trigger-Efficiency}.

As the global trigger rate is very sensitive to the individual channel singles rates, these are readout and recorded in the DAQ at $\sim 0.5 \hertz$. Due to the differences in the response between channels the thresholds are set individually, with different thresholds for each antenna to achieve the desired contribution to the global trigger rate. These thresholds, informed by singles rates, were regularly adjusted to ensure that the system triggered near to the desired rate during operation in 2011-2012. In future stations it is intended that the adjustment of thresholds will be handled by a feedback loop, or servo, which would monitor movements in the singles rates and adjust the thresholds accordingly to facilitate autonomous operation of the stations.

There are two types of trigger that cause event readout in the TestBed: `RF triggers' which are those described above, and `software triggers' which are taken at approximately $1 \hertz$. Software triggers are instigated by the acquisition software controlling data taking in the DAQ box and are intended to provide a `minimum bias' sample of events that profile the radio environment over the course of the year.

\begin{figure}[htpb]
  \centering
  \includegraphics[width=\largefigwidth]{chapters/ara-detector/Global-Trig-Efficiency.pdf}
  \caption{TestBed trigger efficiency measured as a function of voltage signal to noise ratio (SNR) for an impulsive signal from \cite{Allison2012457}. The SNR is measured with respect to the RMS receiver voltage from baseline thermal noise. The three lines represent the efficiency measurements at different electronically set threshold values for the output of the tunnel diode power detector.}
  \label{fig:ara-detector:TestBed:Trigger-Efficiency}
\end{figure}


\subsection{Digitisation}
\label{sec:ara-detector:TestBed:Digitisation}

Signals are continuously sampled using a switched capacitor array until a trigger condition is met. At this point the FPGA initiates a freeze in the analogue sampling and those samples are digitised and readout, taking about $30 \milli \second$. The sampling and digitisation is performed by 3 functionally identical LABRADOR digitisers designed and produced by the Instrument Design Laboratory at the University of Hawaii \cite{Varner2007447}, and is described in detail in \SectionRef{sec:calibration:LABRADOR-Digitiser-Chip}. The digitisers are operated at a sampling speed of $1 \giga \mbox{Sample} /\second$ giving a Nyquist frequency of $500 \mega \hertz$ and consist of a switched capacitor array which track and sample the input voltage. The sampling speed is low compared with the maximum frequency expected from Askaryan signals, which can reach $1 \giga \hertz$. In order to make use of the high frequency response of VPol and HPol antennas used in the boreholes these channels were split and fed into two digitiser channels each, with a $0.5 \nano \second$ offset between the two. This interleaving leads to an increased effective sampling rate, with samples being taken every $0.5 \nano \second$ and pushing the Nyquist frequency up to $1 \giga \hertz$. In addition the sampled waveforms for these channels are the same length in time as those for other channels and allow for a readout window $> 100 \nano \second$. The net result is that 8 of the antennas deployed in the boreholes are effectively sampled at $2 \giga \mbox{Sample} /\second$, with the remaining antennas sampled at $1 \giga \mbox{Sample} /\second$.

The digitised waveforms consist of 256 samples per channel and are $\sim 256 \nano \second$ long. For the interleaved channels there are 512 samples taken over the same time period. These are readout and packetised by the FPGA before being transferred via a USB connection to software running on a single board computer (SBC).

\subsection{Calibration systems}
\label{sec:ara-detector:TestBed:Calibration-systems}

In order to assess the functionality of the TestBed and provide an impulsive calibration source a calibration pulser was installed along with the TestBed detector. A GPS-synchronised Rubidium clock was used to trigger the calibration pulser that sends a $\sim 250 \pico \second$ duration pulse to antennas deployed $40 \meter$ from the centre of the detector. The output of the calibration pulser was fed into either a VPol or HPol antenna, of the same design as those deployed in the boreholes, illuminating the TestBed at a rate of $1 \hertz$ during data taking. The connection to the transmit antennas requires manual intervention to switch between polarisations, so the pulser was connected to the VPol antenna for the duration of 2011, then switched in the summer season to the HPol antenna for 2012.


\subsection{Data acquisition and data transfer}
\label{sec:ara-detector:TestBed:Data-Acquisition}

\FigureRef{fig:ara-detector:TestBed:DAQ-Schematic} shows a diagram of the TestBed DAQ, which is housed in a RF shielded aluminium box. The DAQ box sits in a wooden coffin buried in a pit under the surface of the ice. The output of the signal chain from each of the receive antennas is attached to a series of connectors on the outside of the DAQ box. A final stage of filters is then applied to remove any out of band noise associated with transmission from the boreholes to the surface, before splitting into the digitisation and triggering chains and insertion to the TestBed electronics.

\begin{figure}[htpb]
  \centering
  \includegraphics[width=\hugefigwidth]{chapters/ara-detector/TB-DAQ-Schematic.pdf}
  \caption{Block diagram of the TestBed DAQ from \cite{Allison2012457}.}
  \label{fig:ara-detector:TestBed:DAQ-Schematic}
\end{figure}


The TestBed makes use of a custom digitisation and trigger board - the IceCube Radio Readout (ICRR), a custom USB readout board and a commercial SBC. The ICRR board is shown in \FigureRef{fig:ara-detector:TestBed:ICRR}.

\begin{figure}[htpb]
  \centering
  \includegraphics[width=\largefigwidth]{chapters/ara-detector/ARA_Electronics_ICRR.jpg}
  \caption{The ICRR board from \cite{2011ICRC....4..350A}. On the left the 16 digitisation chain inputs, right the 16 trigger inputs and centre the FPGA. The boards underneath are no longer used in the TestBed.}
  \label{fig:ara-detector:TestBed:ICRR}
\end{figure}

The TestBed station, and future stations, have their data taking and operation managed by dedicated software running on a SBC installed in the DAQ box. Due to the extreme conditions in Antarctica these SBCs must be cold tested in the northern hemisphere (as with all of the electronics deployed) prior to deployment to ensure that they are able to function properly at very cold temperatures (the ambient temperature at the south pole falls below $-80 \celsius $ during the winter months).

The data acquisition software is tasked with starting and stopping  periods of data taking, or runs, periodically requesting software triggered events, reading out digitised events, reading housekeeping and sensor information and storing all of this data locally. The housekeeping and sensor data consists of useful information to asses the condition of the station and trigger, such as the single antenna trigger rates and ambient temperature. All of the data recorded by the SBC is temporarily stored on a flash drive before being transferred to a dedicated server in the IceCube Laboratory. Data transfer occurs over twisted pair wires via an ethernet modem, which is then archived on the receiving server. Approximately $\sim 10 \%$ of the data is selected to be sent north via a satellite link to provide a sample to be monitored to asses the detectors' functionality during the course of the year. The full data set was carried by hand on hard disks to a data warehouse in the north during each summer season, when it is possible to move personnel and equipment in and out of the South Pole Station. 


\section{ARA 1-3}
\label{sec:ara-detector:ARA1-3}

The first full station, ARA1, was deployed in 2011-2012 at a depth of 100 \meter, with two further stations ARA2 and ARA3 deployed during the 2012-2013 season at a depth of 200 \meter. The updated design follows that illustrated in \FigureRef{fig:ara-detector:ARA-37:ARA-37} with 4 boreholes each containing 4 receive antennas, and a further 2 boreholes containing transmit antennas to illuminate the station with calibration signals. The updated specifications for the stations are summarised in \TableRef{tab:ara-detector:ARA1-3:Specifications} and can be compared with those in \TableRef{tab:ara-detector:TestBed:Specifications}.

\begin{table}
\begin{center}
  \begin{tabular}{ p{0.48\textwidth} p{0.48\textwidth} }
    \textbf{Specified Parameter}  & \textbf{Full Station} \\    
    \hline
    Number of VPol antennas                            & 8\\
    VPol antenna type                                  & bicone\\
    VPol antenna bandwidth (\mega\hertz)               & 150-850 \\
    Number of HPol antennas                            & 8 \\
    HPol antenna type                                  & QSC \\
    HPol antenna bandwidth (\mega\hertz)               & 200-800\\
    Surface antenna type                               & fat dipole\\
    Surface antenna bandwidth (\mega\hertz)            & 30-300 \\
    Number of surface antennas                         & 4  \\
    Number of receive antenna boreholes                & 4  \\
    Borehole depth (\meter)                            & 200\\
    Vertical antenna configuration                     & VPol (HPol) above HPol (VPol)    \\
    Vertical spacing (\meter)                          & 20\\
    Approximate geometry                               & trapezoidal\\
    Approximate radius (\meter)                        & 10\\
    Number of calibration antenna boreholes            & 2\\
    Calibration borehole distance from centre (\meter) & 30\\
    Calibration hole geometry                          & facing two sides\\
    Calibration signal type                            & impulse and noise\\
    LNA noise figure (\kelvin)                         & $<80$ \\
    LNA/amplifier dynamic range                        & 30:1\\
    RF amplifier total gain (dB)                       & $>75$\\
  \end{tabular}
  \caption{ARA1-3 detector specifications.}
  \label{tab:ara-detector:ARA1-3:Specifications}
\end{center}
\end{table}


The successful deployment of antennas at much greater depth than the TestBed greatly reduces the ray-bending problems caused by the changing index of refraction in the firn layer of ice (approximately the first $\sim 150 \meter$ of ice below the surface). This success, however, necessitates a different method of transmitting signals from down-hole to the surface, since the losses over these distances are large using RF coaxial cables. The output of the receive antennas are passed through the initial filters and LNAs as in the TestBed. The output of the LNAs undergo a further stage of amplification before being converted into an optical signal. This optical signal is then transmitted to the surface over fiber cables, which suffer from much lower losses and noise insertion levels than coaxial RF cables. Each borehole, hosting 4 receive antennas, has 1 Down-hole Transmission Module (DTM) unit that performs the RF to optical conversion and amplification. At the surface the signal is converted back to RF via a Fiber Optical Amplification Module (FOAM), which also further boosts the signal with $\sim 40 \mbox{dB}$ of amplification, before the insertion into the DAQ box.

A new DAQ system was designed and implemented for the full ARA stations based on the use of a new digitiser, again designed and supplied by the Instrument and Design Laboratory at the University of Hawaii. The Ice Ray Sampler (IRS2) application specific integrated circuit (ASIC) was designed to address the design goals of ARA. The IRS2 features higher sampling speeds across all of the input channels ($3.2 \giga \mbox{Sample} /\second$) to make use of the high frequency response of the deep antennas and to better resolve the high frequency content in Askaryan signals. The IRS2 also features a large analogue sample buffer, approximately $10 \micro \second$ long, which allows for increased trigger time windows than the LABRADOR digitiser used in the TestBed. This is an important feature since the transit time of RF signals across the detector increases significantly from the TestBed to the new stations, due to the increased distance between receive antennas. This buffer is also deep enough to enable future multi-station triggers, which could be used to further determine event topologies.

The triggering systems were improved to offer finer adjustment of threshold levels for the individual antennas that contribute to RF triggers. For the TestBed the threshold tuning was a time consuming process performed by a user over low bandwidth satellite links to the South Pole Station. The increased granularity of threshold setting was used to implement automatic threshold setting via a feed-back, or servo, loop. The individual antenna trigger rates are read out by the DAQ at $1 \hertz$ and used to inform new threshold settings. The continuous adjustment of the thresholds means that the DAQ is able to operate with long-term stability of RF trigger rate, without the need for regular user intervention.

The calibration systems were also changed significantly. Two calibration pulsers were deployed, one in each of the calibration boreholes. Each of the two pulsers are attached to a pair of antennas, 1 VPol and 1 HPol, providing 4 antennas with which to illuminate each station. The DAQ is linked to these pulsers via communications cables that allow switching between the output antennas and attenuation of the pulser output. These features were integrated into the DAQ software to allow calibration pulser settings to be set at the start of each run of data taking, allowing much greater flexibility than in the TestBed, where the setup could only be changed by physical means during the austral summer.


Details of work undertaken in developing and installing the updated DAQ systems for ARA1-3 are discussed in \SectionRef{sec:calibration:ARA1-3-development}.

%Work undertaken for this PhD included:
%
%\begin{itemize}
%\item Testing of the bespoke electronics boards that host the FPGA and sub-systems
%\item Firmware for a USB micro-controller used for data readout and communication between SBC and FPGA
%\item Testing and characterising a number of iterations of the IRS ASIC
%\item Writing and developing the DAQ software
%\item Testing and developing the triggering systems
%\item Writing and implementing the event readout format
%\item Installation and commissioning of the DAQ on site in Antarctica for ARA2 and ARA3
%\item Remote detector operation and maintenance for ARA1-3 during 2012
%\end{itemize}
%





%\begin{table}
%\begin{center}
%  \begin{tabular}{ p{0.4\textwidth} | p{0.2\textwidth} | p{0.2\textwidth}}
%    \textbf{Specified Parameter} & \textbf{TestBed Station} & \textbf{Full Station} \\    
%    \hline
%    Number of VPol antennas                            & 2 near-surface, 4 in ice           & 8\\
%    \hline
%    VPol antenna type                                  & bicone                             & bicone\\
%    \hline
%    VPol antenna bandwidth (\mega\hertz)               & 150-850                            & 150-850 \\
%    \hline
%    Number of HPol antennas                            & 2 near-surface, 6 in ice           & 8 \\
%    \hline
%    HPol antenna type                                  & bowtie-slotted-cylinder            & quad-slotted cylinder \\
%    \hline
%    HPol antenna bandwidth (\mega\hertz)               & 250-850                            & 200-800\\
%    \hline
%    Surface antenna type                               & fat dipole                         & fat dipole\\
%    \hline
%    Surface antenna bandwidth (\mega\hertz)            & 30-300                             & 30-300 \\
%    \hline
%    Number of surface antennas                         & 2                                  & 4  \\
%    \hline
%    Number of receive antenna boreholes                & 4                                  & 4  \\
%    \hline
%    Borehole depth (\meter)                            & 30                                 & 200\\
%    \hline
%    Vertical antenna configuration                     & VPol (HPol) above HPol (VPol)      & VPol (HPol) above HPol (VPol)    \\
%    \hline
%    Vertical spacing (\meter)                          & 5                                  & 20\\
%    \hline
%    Approximate geometry                               & trapezoidal                        & trapezoidal\\
%    \hline
%    Approximate radius (\meter)                        & 10                                 & 10\\
%    \hline
%    Number of calibration antenna boreholes            & 3                                  & 2\\
%    \hline
%    Calibration borehole distance from center (\meter) & 30                                 & 30\\
%    \hline
%    Calibration hole geometry                          & equilateral triangle               & facing two sides\\
%    \hline
%    Calibration signal type                            & impulse only                       & impulse and noise\\
%    \hline
%    LNA noise figure (\kelvin)                         & $< 80$                             & $<80$ \\
%    \hline
%    LNA/amplifier dynamic range                        & 30:1                               & 30:1\\
%    \hline
%    RF amplifier total gain (dB)                       & $>75$                              & $>75$\\
%  \end{tabular}
%  \label{tab:ara-detector:TestBed:Specifications}
%\end{center}
%\end{table}
%
