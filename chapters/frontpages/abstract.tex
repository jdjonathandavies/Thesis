The Askaryan Radio Array (ARA) is a new experimental effort to develop an array of sub-detectors capable of measuring ultra-high energy neutrino-induced radio pulses in the Antarctic ice sheet. Each sub-detector is able to function as a stand alone neutrino detector, the first of which was installed during the 2011 austral summer. In the following two years a further 3 sub-detectors were installed with updated design and functionality, with more planned over the next few years.

This thesis will describe an analysis of the data collected by the first ARA station and presents the results of a search for ultra-high energy neutrinos. No statistically significant evidence for neutrino-induced signals is observed, with no candidate neutrino events. A limit is placed on the flux of ultra-high energy neutrinos and extrapolated to the full 37 station ARA detector operated over a 5 year period.
