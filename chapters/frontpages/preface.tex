You might want something generic about your experiment here and an overview of your work here.

The Large Hadron Collider (\LHC) is the largest and highest energy particle accelerator in the world, designed to collide protons with an unprecedented centre-of-mass energy of \unit{14}{\TeV} and instantaneous luminosity of \highL.
In addition, the \LHC has a heavy ion collision programme, aiming to collide lead nuclei with a centre-of-mass energy of \unit{5.5}{\TeV}.
In the early phase of operation, the proton-proton programme at the \LHC has been operating with reduced centre-of-mass energies of up to \unit{7}{\TeV}; these nevertheless represent the highest energy collisions that have yet been attained in a particle accelerator.

In this thesis, a number of separate analyses are presented, each aiming to probe our understanding of \QCD in this new energy regime, with $\rootS = \unit{7}{\TeV}$.
Differential \xs{s} of inclusive jets and \dijet{s} are performed across two orders of magnitude in jet transverse momentum and \dijet mass and are compared to next-to-leading order theoretical predictions.
Radiation between \dijet{s} is examined as a possible means of discriminating between DGLAP and \BFKL-like parton evolution schemes.

Finally, a more technical contribution to this thesis is a technique for ascertaining the uncertainty on the jet energy scale through intercalibration between jets in different regions of the detector.
