\chapter{Calibration}
\label{chap:Calibration}

Askaryan radiation from UHE neutrino-induced particle cascades results in short duration bipolar signals. These impulses typically extend over $\sim 100 \pico \second$ and result from coherent emission of frequencies up to $500 \mega \hertz$ to $1 \giga \hertz$. In order to make measurements of such signals it is not only necessary to use antennas with the necessary bandwidth, but also to be able to sample and digitise their output with an appropriate fidelity. 

The strict power and cost requirements of ARA necessitate the use of application specific integrated circuit (ASIC) digitisers that fulfill these criteria. The majority of recorded signals that trigger ARA stations are expected to be from thermal or anthropogenic sources and are considered backgrounds to detection of neutrino signals. Timing differences between the same signal being observed in multiple antennas can be used to reconstruct the physical location of their source. In \SectionRef{chap:Analysis} it will be shown that reconstruction can provide a powerful tool with which to separate interesting signals from these backrounds. 

\begin{figure}[htpb]
  \includegraphics[width=\largefigwidth]{chapters/calibration/LAB3_archictecture.pdf}
  \caption{\cite{Varner2007447}}
\end{figure}

\begin{figure}[htpb]
  \includegraphics[width=\largefigwidth]{chapters/calibration/LAB3_RCO_Schmatic.pdf}
  \caption{\cite{Varner2007447}}
\end{figure}
