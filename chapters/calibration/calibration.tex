\chapter{Calibration and data acquisition development}
\label{chap:Calibration}

 Development and calibration of the data acquisition systems is a crucial part of building a detector sensitive to neutrino signals that could make the first measurements of ultra high energy (UHE) neutrino interactions in the antarctic ice. Askaryan radiation from UHE neutrino-induced particle cascades results in short duration radio frequency (RF) signals. These impulses typically extend over $< 100 \pico \second$ and result from coherent emission of frequencies up to $1 \giga \hertz$. In order to make measurements of such signals it is not only necessary to use antennas with the necessary bandwidth, but also to be able to sample and digitise their output with appropriate fidelity.


This section will describe the need for, and details of, a number of timing calibrations that were performed on the digitisation chain in the TestBed. In addition to optimising the TestBed detector for data analysis, work was undertaken in developing and installing the first 3 full ARA stations. These stations will eventually form part of a much larger array of sub-detectors that is projected to reach world leading sensitivity to UHE neutrino fluxes.

\section{TestBed calibration}
\label{sec:calibration:TestBed-calibration}



\subsection{TestBed timing calibrations}
\label{sec:calibration:LABRADOR-Digitiser-Chip}

The majority of recorded signals that trigger the ARA stations are expected to be from thermal or anthropogenic sources and are considered backgrounds to detection of neutrino signals. Timing differences between signals being observed in multiple antennas can be used to reconstruct the physical location of their source. In \SectionRef{chap:Analysis} it will be shown that reconstruction can provide a powerful tool with which to separate interesting signals from these backgrounds. The need to reject backgrounds in this manner combined with the signal shape expected from Askaryan radiation sets a requirement for large bandwidth, high sampling rate digitisation of the signals received in the detector's antennas. The remote location and associated power restrictions preclude the use of commercially available digitisers and necessitate the use of an application specific integrated circuit (ASIC) digitiser.

The TestBed digitises waveforms using three LABRADOR (Large Analogue Bandwidth Recorded And Digitizer with Ordered Readout) ASIC chips designed at the Instrument and Design Laboratory, University of Hawaii \cite{Varner2007447}. Each chip has 9 parallel RF input channels that are continuously sampled. These samples are digitised in place once read out is initiated by the data acquisition (DAQ). Sampling, digitisation and readout of samples is controlled by signals from a field programmable gate array (FPGA) in the DAQ. A number of timing related calibrations are necessary due to the properties of the LABRADOR chip and its implementation in the TestBed. 

\begin{figure}[htpb]
  \includegraphics[width=\hugefigwidth]{chapters/calibration/LAB3_archictecture.pdf}
  \caption{The architecture of the LABRADOR digitiser from \cite{Varner2007447}. 9 RF input channels are sampled in parallel using a common timing control.}
  \label{fig:calibration:LABRADOR-Digitiser-Chip:Architecture}
\end{figure}

\FigureRef{fig:calibration:LABRADOR-Digitiser-Chip:Architecture} shows the architecture of the LABRADOR chip used in the TestBed. On each chip a switched capacitor array (SCA) of 260 sampling capacitors per channel is used to sample the input voltages. Sampling is controlled by a write signal, which is common to all 9 channels, driven by an externally provided clock. When a stop signal is asserted by the DAQ the samples are digitised in place using a Wilkinson analogue to digital converter (ADC), which occurs in parallel for all samples on all channels. Once digitisation is complete the digitised voltage values are requested by the DAQ and read out in sequence on a 12-bit data bus. The FPGA then process these values for packetisation and recording as an event for later analysis.

\begin{figure}[htpb]
  \includegraphics[width=\hugefigwidth]{chapters/calibration/LAB3_RCO_Schmatic.pdf}
  \caption{A schematic of the sampling within the LABRADOR digitiser from \cite{Varner2007447}.}
  \label{fig:calibration:LABRADOR-Digitiser-Chip:Schematic}
\end{figure}


To achieve continuous sampling, voltage samples are stored in the SCA as the write signal propagates along the array. The time taken for the write signal to propagate between adjacent samples varies from sample to sample and from digitiser to digitiser. These `inter-sample' times are estimated from known input signals and a calibration applied to correct the time base of recorded waveforms. The voltage sampling is illustrated in \FigureRef{fig:calibration:LABRADOR-Digitiser-Chip:Schematic}. The write pointer signal propagates along the array in one of two phases, Ripple Carry Out (RCO) 0 and RCO1. The transit time of the write signal between adjacent samples differs between these two phases, hence calibrated inter-sample times must be calculated for each.

In order to continue sampling once the write signal reaches the end of the SCA the write pointer wraps back to the first sample, at which time the RCO phase changes. This wrapping means the SCA functions as a ring, continuously sampling voltage values and over-writing the oldest sample taken. When a stop request is received, to initiate digitisation and read out, the chip issues a `HITBUS' causing the chip to stop sampling. The HITBUS removes around 8 samples from the waveform as the stored values are inaccurate due to the changing state of the chip as they are latched. The capacitor immediately after the HITBUS is the earliest sample in the resulting waveform, and the capacitor immediately before the HITBUS is the latest sample. 

Although the physical distance travelled when wrapping the write pointer is small the need to change phase of the write pointer causes a much larger time delay than between samples so, to avoid time gaps in the resulting waveforms, the write pointer begins wrapping at the $255^{\mbox{th}}$ sample. The remaining 4 tail samples, 256-259, are then used to fill in the resulting gap in time between the $255^{\mbox{th}}$ and $0^{\mbox{th}}$ samples.  The time taken for the write pointer to wrap, referred to as the `wrap-around', is another quantity that needs to be estimated and accounted for in calibration, which again varies from chip to chip and between RCO phases.


One of the features of the LABRADOR chip is the ability to operate at a range of sampling speeds. For the TestBed implementation it is operated at a nominal sampling speed of $1 \giga \mbox{Sample} /\second$ which corresponds to a Nyquist frequency \footnote{The Nyquist frequency for a discretely sampled system indicates the maximum resolvable input frequency and is equal to half the sample rate.} of $500 \mega \hertz$. The borehole antennas have good response to frequencies up to $850 \mega \hertz$, which means that frequency content would be lost if these antennas had their signals input to a single channel each. In order to retain this high frequency content each of these antennas has its signal split and inserted into a pair of input channels, one offset from the other by $0.5 \nano \second$. This interleaving process leads to an effective doubling of the sampling speed for the borehole antennas, however the offset needs to be calibrated out in order to accurately reconstruct the original input waveforms. 

As well as ensuring the timing information, on a per antenna basis, is as accurate as possible it is also necessary to correct for any offsets between channels. Due to the use of a common write signal each of the channels on a single chip will be aligned once the inter-sample, wrap-around and interleave calibrations are carried out. This is not true for channels on different chips and must be addressed in the final timing calibration, inter-chip jitter removal. Since timing differences between similar signals in pairs of antennas serves as the basis of event reconstruction this calibration is an integral part of distinguishing between noise sources and potential neutrino induced signals. 


\subsubsection{Inter-sample times}
\label{sec:calibration:LABRADOR-Digitiser-Chip:Inter-sample-times}

The inter-sample times correspond to the time taken for the LABRADOR digitiser's write pointer to propagate between samples. By adjusting control voltages to the chip the global sampling speed can be adjusted, which acts by effecting a change to each of the inter-sample times. Using the selected operating parameters of the chips in the TestBed DAQ a series of dedicated calibration runs were taken in the northern hemisphere prior to deployment in Antarctica. For each of these runs, each containing many thousands of events,  precise frequency sinusoidal inputs were provided to the DAQ box were taken and used to calculate the various timing calibrations necessary for the TestBed. 

The inter-sample times were calculated by measurements of sine waves input to each RF channel on the chip, using a method informed from \cite{AbbyThesis}. The number of zero-crossings between a pair of adjacent samples was scaled by the number of waveforms in which that sample was present. The resulting fractional zero-crossing occupancy is then scaled by half the period of the input sine wave to recover the inter-sample time:

\begin{itemize}
\item The voltage values of each waveform are zero-meaned 
\item The RCO phase of each pair of adjacent samples is then determined as described in \SectionRef{sec:calibration:LABRADOR-Digitiser-Chip:RCO-phase-determination}
\item The total number of times zero is crossed between each pair of samples is counted over the run (i.e. the voltage goes from positive to negative, or visa versa, between a pair of samples)
\item The total number of zero-crossings is then divided by the total number of events for which that pair of samples were active
\item This fractional zero-crossing occupancy is then scaled by the half-period of the input sine-wave, resulting in an estimate for the inter-sample times
\end{itemize}


The resulting distribution of inter-sample times is shown in \FigureRef{fig:calibration:LABRADOR-Digitiser-Chip:Inter-sample-times}. 

\begin{figure}[htpb]
  \subfloat[All RCO phases]{\includegraphics[width=0.49\textwidth]{chapters/calibration/TB-LAB3-DT-All.pdf}}\\
  \subfloat[RCO 0]{\includegraphics[width=0.49\textwidth]{chapters/calibration/TB-LAB3-DT-RCO0.pdf}}\hfill
  \subfloat[RCO 1]{\includegraphics[width=0.49\textwidth]{chapters/calibration/TB-LAB3-DT-RCO1.pdf}}
  \caption{Calculated inter-sample times for all 3 TestBed LABRADOR digitisers. In (a) both RCO phases are included, whereas (b) and (c) show the distributions for RCO phase 0 and 1 respectively.}
  \label{fig:calibration:LABRADOR-Digitiser-Chip:Inter-sample-times}
\end{figure}

\subsubsection{RCO phase determination}
\label{sec:calibration:LABRADOR-Digitiser-Chip:RCO-phase-determination}

A complication to the calibration procedure is that the RCO phase of the LABRADOR chip at the time of a trigger being asserted is not always recorded correctly. When a trigger occurs and a stop request is sent from the DAQ to the LABRADOR digitisers the RCO phase at this point in time is read out along with the event. In the case that the latest sample in the waveform is in the first 20 samples in the SCA the RCO phase was found to be inaccurate. The correct phase is assigned by looking at the average period of a externally provided clock that is inserted into 1 of the 9 RF inputs on each chip. The average period of the clock is calculated for both possible RCO phases, and the one that is closest to the nominal period is selected and assigned to the waveforms digitised by each chip. This is possible since the average sampling speed differs by $\sim 8 \%$ between the two RCO phases.


\subsubsection{Wrap-around times}
\label{sec:calibration:LABRADOR-Digitiser-Chip:Wrap-around-times}

The time taken to wrap between the end and beginning of the SCA, denoted $\varepsilon_{0}$ and $\varepsilon_{1}$ for the two RCO phases, is considerably larger than the inter-sample times. Rather than measuring the scaled zero-crossing occupancy, as described in \SectionRef{sec:calibration:LABRADOR-Digitiser-Chip:Inter-sample-times}, sine waves are fitted to the waveform before and after the wrap-around. By aligning the phase of the two fitted sine waves it is possible to estimate the wrap-around time. The full procedure is as follows:

\begin{itemize}
\item The voltage values of each waveform are zero-meaned
\item Only events containing at least 20 samples before and after the wrap-around are selected
\item Calibrated inter-sample times, calculated as described in \SectionRef{sec:calibration:LABRADOR-Digitiser-Chip:Inter-sample-times}, are applied to the waveform 
\item Sine waves are fitted to the last 20 samples before the wrap-around, and first 20 samples after
\item The fitted sine wave after the wrap-around is moved in time until the phase matches that before the wrap-around
\item The wrap-around time is then taken to be the time between sample 255 and sample 0, and is averaged over a calibration run
\end{itemize}


\begin{figure}[htpb]
  \subfloat[RCO 0]{\includegraphics[width=0.49\textwidth]{chapters/calibration/Wrap-Around-Chip1-RCO0.pdf}}\hfill
  \subfloat[RCO 1]{\includegraphics[width=0.49\textwidth]{chapters/calibration/Wrap-Around-Chip1-RCO1.pdf}}
  \caption{Estimated wrap-around times for all events in a calibration run for TestBed LABRADOR chip 0 and both RCO phases. The average value is taken over all events in the run.}
  \label{fig:calibration:LABRADOR-Digitiser-Chip:Wrap-around}
\end{figure}


\subsubsection{Interleaving}
\label{sec:calibration:LABRADOR-Digitiser-Chip:Interleaving}

To increase the effective sampling rate whilst maintaining the time window of the borehole antennas the signals from each are fed into two digitiser channels with one delayed relative to the other by $\sim 0.5 \nano \second$ (approximately half a sample). The resulting waveforms are then produced by interleaving the two digitised channels for each of these antennas in the offline software. The 8 deep antennas have their signals split in this way across pairs of channels on the first 2 chips. The third chip is operated in the normal manner with each channel supplied by a single antenna.

A calibration is performed to estimate the offset between each pair of channels that have their signals split, such that the resulting interleaved waveforms are as accurate as possible. This is achieved by attempting to align the phase of input sine waves between these pairs of channels as follows:

\begin{itemize}
\item The voltage values of each waveform are zero-meaned
\item Inter-sample and wrap-around calibrations are applied to each channel's waveforms
\item A sine wave is fitted to each pair of split channels
\item The phase offset between each pair of fitted sine waves is recorded
\item The interleave offset for a pair of channels is taken to be the mean phase offset over a run of calibration data
\end{itemize}

The interleave offsets for two pairs of channels are shown in \FigureRef{fig:calibration:LABRADOR-Digitiser-Chip:Interleave}. The calculated offsets are close to half the average inter-sample times resulting in an increase in Nyquist cut-off from $500 \mega\hertz$ to $1\giga\hertz$ for the affected channels.

\begin{figure}[htpb]
  \subfloat[Chip 1 Pair 1]{\includegraphics[width=0.49\textwidth]{chapters/calibration/Interleave-Chip1-Pair1.pdf}}\hfill
  \subfloat[Chip 1 Pair 2]{\includegraphics[width=0.49\textwidth]{chapters/calibration/Interleave-Chip1-Pair2.pdf}}
  \caption{Calculated offsets between 2 pairs of channels on LABRADOR chip 1 in the TestBed.}
  \label{fig:calibration:LABRADOR-Digitiser-Chip:Interleave}
\end{figure}



\subsubsection{Inter-chip jitter correction}
\label{sec:calibration:LABRADOR-Digitiser-Chip:Inter-chip-jitter-correction}

The three LABRADOR digitiser chips used in the TestBed are operated in parallel using the same control signals. Each of the 3 digitisers has a clock signal input into one RF channel that is digitised along with the inputs from the signal chain. These clock signals are used to remove event-to-event jitter of the trigger signals that start digitisation and readout of the SCAs. Using the clock channel it is possible to estimate the phase of the clock signal in each LABRADOR chip. A timing correction is calculated to align the phase of the clocks across all three digitisers and is then applied to all waveforms.

\begin{figure}[htpb]
  \includegraphics[width=\largefigwidth]{chapters/calibration/lagAlagBAllNewCuts_updated_labels_resize_regions_labelled.pdf}
  \caption{The measured clock channel phase relative to the first sample in LABRADOR chips 1 and 2. Events in region A are interpreted as needing $25 \nano \second$ (1 clock period) added to the phase in chip 2, in region D $25 \nano \second$ added to the phase in chip 1 and in region C $25 \nano \second$ added to the phase in both chips 1 and 2.}
  \label{fig:calibration:LABRADOR-Digitiser-Chip:Swiss-flag}
\end{figure}

In \FigureRef{fig:calibration:LABRADOR-Digitiser-Chip:Swiss-flag} the measured phase of the clock channel, relative to the first sample, in two LABRADOR chips are plotted against each other. A feature of the TestBed implementation of the LABRADOR digitiser is that there are regions in this parameter space that are under-populated. This feature is used to resolve period ambiguity between the two chips which would otherwise result in an additional $\pm 25 \nano \second$ (1 clock period) offset between the time base in waveforms recorded in different LABRADOR chips.



\subsubsection{Calibration results}
\label{sec:calibration:LABRADOR-Digitiser-Chip:Calibration-results}

In \SectionRef{sec:Analysis:Reconstruction} a method will be described for reconstruction which is reliant upon precise measurements of the timing offsets between the same signal being observed in multiple antennas. The cumulative effect of the timing calibrations described can be seen in \FigureRef{fig:calibration:LABRADOR-Digitiser-Chip:Timing-Differences} where timing offsets between a pair of antennas are calculated using correlation techniques.

\begin{figure}[htpb]
  \subfloat[Calibrated waveforms]{\includegraphics[width=0.49\textwidth]{chapters/calibration/deltaT-VPol-Ant1-4-Calibrated.pdf}}\hfill
  \subfloat[Uncalibrated waveforms]{\includegraphics[width=0.49\textwidth]{chapters/calibration/deltaT-VPol-Ant1-4-Uncalibrated.pdf}}
  \caption{Timing offsets for a month of calibration pulser events for (a) timing calibrated and (b) uncalibrated waveforms. The antennas used are vertically polarised (VPol) receiving signals from a calibration pulser connected to a VPol transmit antenna buried close to the TestBed detector.}
  \label{fig:calibration:LABRADOR-Digitiser-Chip:Timing-Differences}
\end{figure}

The timing calibration with the largest affect is the inter-chip jitter removal, which corrects for offsets up to $25 \nano \second$ between waveforms recorded on different LABRADOR chips. A shift in the measured mean time offset between the antennas' waveforms is caused by application of cable delays, which account for the differing lengths of cable that signals must propagate through when travelling from the receive antenna to the digitiser. After application of all timing calibrations the resolution on time offsets between antennas is $\sim 100 \pico \second$.


\subsection{Voltage reversal}
\label{sec:calibration:Voltage-reversal}

Close to finalisation of the analysis described in \ChapterRef{chap:Analysis} a reversal of voltage values recorded in two of the horizontally polarised (HPol) antennas' waveforms was identified. This reversal is illustrated in \FigureRef{fig:calibration:Voltage-Reversal} where hundreds of calibration pulser signals originating from the HPol calibration pulser are averaged to remove thermal noise. There is a clear inversion of voltage values for antennas 3 and 4 compared with antennas 1 and 2. It is believed that this flip originates from installation of the antennas upside down relative to antennas 1 and 2, or from inversion of voltages during amplification in the signal chain. 

\begin{figure}[htpb]
  \subfloat[Antenna 1]{\includegraphics[width=0.49\textwidth]{chapters/calibration/avWaveformHPolAnt1.pdf}}\hfill
  \subfloat[Antenna 2]{\includegraphics[width=0.49\textwidth]{chapters/calibration/avWaveformHPolAnt2.pdf}}\\
  \subfloat[Antenna 3]{\includegraphics[width=0.49\textwidth]{chapters/calibration/avWaveformHPolAnt3.pdf}}\hfill
  \subfloat[Antenna 4]{\includegraphics[width=0.49\textwidth]{chapters/calibration/avWaveformHPolAnt4.pdf}}
\caption{Averaged waveforms from HPol borehole antennas. The signal captured is from the HPol calibration pulser operated in 2012. There is a clear inversion of voltage values in antennas 3 and 4 compared with antennas 1 and 2.}
\label{fig:calibration:Voltage-Reversal}
\end{figure}


Checks were made on the vertically polarised (VPol) antenna's waveforms from VPol calibration pulser events. No such inversion was found for these antennas. A correction was applied in the analysis described in \ChapterRef{chap:Analysis} that switches positive for negative (and visa versa) voltages for the affected HPol antennas, resulting in a significant improvement in the reconstruction of calibration signals. The affect on measured timing offsets for calibration pulser waveforms is shown in \FigureRef{fig:calibration:Voltage-Reversal-DeltaT}, where the application of a correction for the voltage flip results in the removal of a second peak in the distribution.

\begin{figure}[htpb]
  \subfloat[With voltage reversal correction]{\includegraphics[width=0.49\textwidth]{chapters/calibration/DeltaT-HPol-1-4-Flip.pdf}}\hfill
  \subfloat[Without voltage reversal correction]{\includegraphics[width=0.49\textwidth]{chapters/calibration/DeltaT-HPol-1-4.pdf}}
\caption{The offset between HPol calibration pulser signals recorded in HPol antennas 1 and 4. The offset is measured by taking the time-offset corresponding to maximum correlation between the received signals. The offset is shown with (a) and without (b) correction for the voltage reversal found in antennas 3 and 4.}
\label{fig:calibration:Voltage-Reversal-DeltaT}
\end{figure}



\section{ARA 1-3 DAQ development}
\label{sec:calibration:ARA1-3-development}

During this PhD work was carried out developing the DAQ for the first 3 full ARA stations, ARA 1-3. The DAQ installed for the TestBed differs significantly from that developed and deployed for ARA 1-3. The most prominent change is the use of a new digitiser, the Ice Ray Sampler (IRS2) \cite{IRS2}, in place of the LABRADOR digitiser. The IRS2 features increased sampling speed, up to $3.2 \giga \mbox{Sample}/\second$, and a large analogue sample depth of $\sim 10 \micro \second$. These features mean that higher fidelity waveforms can be recorded and trigger conditions can be made on a much longer time-scale than in the TestBed. To accommodate the new digitiser a new suite of electronics boards were also designed and developed, along with associated firmware and software to manage data taking and event read out.

\subsection{IRS2 testing and calibration}
\label{sec:calibration:ARA1-3-development:IRS2}

 The IRS2 provides 8 radio frequency (RF) inputs and is operated at a sampling speed of $3.2 \giga \mbox{Sample} /\second$, placing the Nyquist frequency well above the maximum anticipated from neutrino induced signals. In addition to the increased sampling speed the IRS2 features analogue buffering of sampled signals. The IRS2 contains 2 sets of 64 capacitor sampling cells (referred to as `blocks') along with 512 analogue storage blocks for buffering sampled signals. Once every $20 \nano \second$ one of the 2 blocks of sampling cells is transferred to the storage buffer whilst sampling of the RF signal continues in the other block. When a trigger condition is met the relevant analogue storage blocks are requested, digitised and read out, whilst sampling and buffering to the remaining storage blocks continues. 

A number of iterations of the IRS digitiser were tested in the laboratory to ascertain their functionality and suitability to use in the new ARA stations. The revision used in ARA 1-3, the IRS2B, was selected as the best functioning among the chips tested. Along with finding, setting and testing the optimal operating parameters for the IRS digitiser preliminary calibration work was undertaken to correct the time base of the waveforms digitised in the IRS2 chip. The results of the latest iteration of timing calibrations are illustrated in \FigureRef{fig:calibration:ARA1-3-development:DDA-Waveforms}, where an uncalibrated and calibrated waveform are shown for a sine wave input into one of the IRS2 channels. 

\begin{figure}[htpb]
  \subfloat[Uncalibrated waveform]{\includegraphics[width=0.49\textwidth]{chapters/calibration/DDA1_Channel6_NoCalibration.pdf}}\hfill
  \subfloat[Calibrated waveform]{\includegraphics[width=0.49\textwidth]{chapters/calibration/DDA1_Channel6_Calibrated.pdf}}
  \caption{The results of applying the latest iteration of timing calibrations to a sine wave input to a channel on the IRS2 chip. The timing calibration was produced by Thomas Meures.}
  \label{fig:calibration:ARA1-3-development:DDA-Waveforms}
\end{figure}

\subsection{Other DAQ work}
\label{sec:calibration:ARA1-3-development:Other-DAQ-work}

Testing and development work was undertaken ahead of deployment for ARA 1-3. Among other activities this included:

\begin{itemize}
\item Development and integration of offline software to accommodate new data formats from the ARA 1-3 DAQ systems
\item Writing the event read out format and development of the online data acquisition software used to control data taking in the DAQ
\item Writing and testing firmware implemented on a USB micro-controller that provides the data connection and control interface between the single board computer, running the online acquisition software, and the field programmable gate array (FPGA) that controls digitisation and triggering in the DAQ
\item Testing and developing the trigger subsystems
\end{itemize}

During the 2012-2013 austral summer I travelled to the South Pole to install and commission the DAQs for ARA 2 and ARA 3. After installation of the new systems in the ice work was undertaken in setting up the data flow from each station to a server located in the IceCube Laboratory. Once data flow was established systems were put in place to allow a subset of data to be transferred via satellite to a data warehouse in the northern hemisphere. 

Upon completion of the installation and deployment activities systems were developed to allow monitoring and controlling of the stations from the north during the course of the austral winter. As part of this work I took responsibility for monitoring and controlling station operations for the duration of 2013, with little down time experienced in ARA 2 and ARA 3 during the year. The event rate between April and September 2013 for ARA 2 is shown in \FigureRef{fig:calibration:ARA1-3-development:ARA2-Rate}. Due to disk space limits the trigger threshold was changed at the beginning of May. The affect of this change was to lower the event rate to a manageable level for the remainder of the year. 

\begin{figure}
  \includegraphics[width=\largefigwidth]{chapters/calibration/ARA02-EventRate.pdf}
  \caption{Event rate in ARA2 during 2013.}
  \label{fig:calibration:ARA1-3-development:ARA2-Rate}
\end{figure}

Due to an electronics issue the DAQ installed at ARA 1 was unable to take data throughout the year and stopped functioning around a month after the summer season closed. The issues encountered with ARA 1 exemplify the challenges faced in developing and maintaining detectors in such a remote location and in such harsh conditions. These problems are tempered by the success of ARA 2 and ARA 3, showing long term stability of the new DAQ systems over the course of 2013.






